\documentclass[prd, nofootinbib, floatfix, 12pt,tightenlines]{revtex4}
%\documentclass[useAMS,usenatbib,tightenlines,11pt,preprint]{aastex}
%testing push


\usepackage[paperwidth=8.5in,paperheight=11in,centering,hmargin=1in,vmargin=1in]{geometry}
\usepackage{amsmath}
\usepackage{amsbsy}

\topmargin0.0cm
\textheight8.5in


\input epsf
\usepackage{amsmath,amssymb,subfigure}
\usepackage{graphicx}
\usepackage{epsfig}
\usepackage{color}
%\usepackage{ulem}
%\usepackage{epstopdf}

\renewcommand{\topfraction}{0.95}
\renewcommand{\bottomfraction}{0.95}





%%%%%%%%%%%%%%%%%%%%%%%%%%%%%%%%%%%%%%%%%%%%%%%%%%%%%%%%%%%%
%%%%%%%%%%%%%%%%%%%%%%%%%%%%%%%%%%%%%%%%%%%%%%%%%%%%%%%%%%%%
%%%%%%%%%%%%%%%%%%%%%%%%%%%%%%%%%%%%%%%%%%%%%%%%%%%%%%%%%%%%

\begin{document} 
\sloppy
\title
{An Active Learning Approach to Optimizing Astronomy
}

%\pagerange{\pageref{firstpage}--\pageref{lastpage}}

\label{firstpage}

% \date{\today}

\maketitle 

%%%%%%%%%%%%%%%%%%%%%%%%%%%%%%%%%%%%%%%%%%%%%%%%%%%%%%%%%%%%
%%%%%%%%%%%%%%%%%%%%%%%%%%%%%%%%%%%%%%%%%%%%%%%%%%%%%%%%%%%%
%\begin{abstract} 

%\end{abstract} 

\section{Introduction}



\section{Photometric Redshifts}

The fact that three quarters of the Universe is composed of something we absolutely
do not understand (dark energy) represents one of the deepest puzzles in fundamental
physics today.  Fortunately, as a fundamental constituent of the cosmos, dark energy
has a strong effect on the formation history of galaxies.
Detailed maps of the three dimensional distribution of galaxies 
\cite{muvarpi2,roland} and statistical studies of how their images are distorted via
gravitational weak lensing \cite{sudeep} can teach us much of what we do not
currently know about the dark sector of the Universe.  These maps will require
robust measures of the cosmological redshift of the light emitted by galaxies.

While it is straightforward to determine a galaxy's redshift by directly taking
its spectrum, this is an expensive measurement.  
Thus, survey astronomers rely on
photometric redshifts -- redshifts inferred from observations of the galaxy
through a handful of broad-band photometric filters.  Given a
galaxy's spectrum and knowledge of the profile of our filters, we
can in principle determine the galaxy's redshift by artificially shifting the
theoretical spectrum, integrating it through the filter, and repeating until we have a
good match for the observed photometry.  Unfortunately, galaxies come in many
varieties and the process is not as simple as the one-parameter fit described. 
Photometric redshift algorithms must fit for all types of galaxies at all
observed redshifts using data from as few as five photometric
filters.  This
multi-parameter fit is highly non-trivial and incorrect calculations can degrade
dark energy constraints by nearly an order of magnitude \cite{Ma2006}.
New algorithms must be designed and tested against high-fidelity simulations of the
forthcoming data in order to realize the results anticipated by the Dark
Energy Task Force \cite{detf}.  


\section{Transient Classification}

In addition to the unknown constituents of cosmology, future surveys promise to
yield observations of unknown sources within our own galaxy.  Most exciting
among these will be the variable sources: stars whose luminosity changes either
periodically or cataclysmically.  Some such sources are already known and
well-understood, such as type II supernovae.  
Some are known, but mysterious, such as the afterglows of
Gamma Ray Bursts.
Many others are still waiting to be discovered \cite{sciencebook}.  As with
cosmological redshift, it would be ideal if we could take detailed spectra and
time-series photometry of each source in order to understand its physical
nature.  This is unrealistic.  The LSST is projected to detect as many as one
million variable sources every night \cite{lsstoverview}. 
There are not sufficient follow-up resources to target every one of these
sources with a detailed observation. 
Astronomers must develop algorithms to identify,
in real time, using sparse observations which transients are novel and which are 
well-understood and what follow-up observations would yield the most information
about the novel sources.  These algorithms must be rapid and robust, as the
decision to follow-up a given transient must be made before the transient fades
back into quiescence.  In addition to their impact on future surveys, such as
the LSST, these algorithms will find immediate application characterizing
observations taken by The Palomar Transient
Factory\footnote{http://www.astro.caltech.edu/ptf/} and Catalina Sky
Survey\footnote{http://www.lpl.arizona.edu/css/}.

Real-time automatic classification of objects is already widely acknowledged as a
necessary support technology for the forthcoming age of survey astronomy
\cite{djorgovski2011,richards2011,richards2012,graham2012,mahabal2008a,mahabal2011a}.
Objects will need to be categorized into known science classes so that novel
or rare objects can be flagged for detailed follow-up observations.
For transient events, algorithms must be able to make rapid
decisions so that sources can be targeted for follow-up 
and classifications learned before
objects return to their quiescent phases.  A great deal of work has
already been done developing algorithms that can learn the classification of an
object given a fixed set of observations and training data.  
Mahabal {\it et al}. (2008a,2011a,2011b) propose to break down the
observations of a given object into $\{\Delta m,\Delta t\}$ pairs (where $m$
is magnitude and $t$ is time) and use the density of observations in this 
two-dimensional space as the basis for a Bayesian classification algorithm.
Mahabal {\it et al}. (2008b) alternatively propose to use those same 
$\{\Delta m,\Delta t\}$ pairs as the input to a Gaussian process regression
by which they will reconstruct the object's entire light curve as a function
of time, and then classify the object based on that reconstruction.
Richards {\it et al}. (2011) use observations of transient objects to extract
periodic (e.g. the amplitude and frequency of the first two Fourier modes of
the object's light curve) and non-periodic (e.g. the variance and skewness of
all of the magnitude observations taken, regardless of their separation in
time) and feed those features into several tree-based classifiers.
They find misclassification rates lower than 30\% with their best method yielding a
misclassification rate of 22.8\%.  Using only non-periodic features, which will be
especially easy for survey telescopes to gather, rather than
full light curves, they 
find a misclassification rate of
between 26\% and 28\%.
Bloom {\it et al}. (2011) also use a tree-based automatic classifier on
Palomar Transient Factory data and find a 3.8\% error rate when
discriminating between four major classifications.  Richards {\it et al}.
consider a more complete set of 25 possible classifications.
Clearly, many possibile approaches are available for the automated
classification of transient objects, and not all of them rely upon highly
detailed observations to function.

\section{On-Line Learning Algorithms}

\newpage
%%%%%%%%%%%%%%%%%%%%%%%%%%%%%%%%%%%%%%%%%%%%%%%%%%%%%%%%%%%%
%%%%%%%%%%%%%%%%%%%%%%%%%%%%%%%%%%%%%%%%%%%%%%%%%%%%%%%%%%%%

%%%%%%%%%%%%%%%%%%%%%%%%%%%%%%%%%%%%%%%%%%%%%%%%%%%%%%%%%%%%
%%%%%%%%%%%%%%%%%%%%%%%%%%%%%%%%%%%%%%%%%%%%%%%%%%%%%%%%%%%%
\begin{thebibliography}{99}

\bibitem[Abazajian {\it et al}. 2009]{Abazajian:2008wr}
  Abazajian, K.~N.{\it et al.}  [SDSS Collaboration]~2009,
  %``The Seventh Data Release of the Sloan Digital Sky Survey,''
  The Astrophysical Journal Supplement Series  {\bf 182}, 543
  [arXiv:0812.0649 [astro-ph]].
  %%CITATION = APJSA,182,543;%%

\bibitem[Abdalla {\it et al}. 2011]{abdalla}
Abdalla,~F.~B., Banerji,~M., Lahav,~O., and Rashkov,~V. 2011,
Monthly Notices of the Royal Astronomical Society {\bf 417}, 1891

\bibitem[Abramo {\it et al}. 2012]{narrow}
Abramo,~L.~R., Strauss,~M.~A., Lima,~M., Hern\'andez-Monteagudo,~C., Lazkoz,~R.,
Moles,~M., de Oliveira,~C.~M., Sendra,~I., Sodr\'e Jr.,~L., and
Storchi-Bergmann,~T. 2012, Monthly Notices of the Royal Astronomical Society
{\bf 423}, 3251

\bibitem[Albrecht {\it et al}. 2006]{detf}
Albrecht,~A., Bernstein,~B., Cahn,~R., Freedman,~W.~L., Hewitt,~J.,
Hu,~W., Huth,~J., Kamionkowski,~M., Kolb,~E., Knox,~L., Mather,~J.~C.,
Staggs,~S., Suntzeff,~N.~B. (Dark Energy Task Force) 2006,
``Report of the Dark Energy Task Force,''
\verb|http://jdem.gsfc.nasa.gov/science/DETF_Report.pdf|


\bibitem[Berg\'e {\it et al}. 2012]{psf}
Berg\'e,~J., Price,~S., Amara,~A., and Rhodes,~J. 2012,
Monthly Notices of the Royal Astronomical Society {\bf 419}, 2356

\bibitem[Bloom {\it et al}. 2011]{bloom2011}
Bloom,~J.~S., Richards,~J.~W., Nugent,~P.~E., Quimby,~R.~M., Kasliwal,~M.~M.,
Starr,~D.~L., Posnanski,~D., Ofek,~E.~O., Cenko,~S.~B., Butler,~N.~R.,
Kulkarni,~S.~R., Gal-Yam,~A., and Law,~N. 2011 [arXiv:1106.5491]

\bibitem[Bonfield {\it et al}. 2010]{Bonfield}
Bonfield,~D.~G., Sun,~Y., Davey,~N., Jarvis,~M.~J., Abdalla,~F.~B.,
Banerji,~M., Adams,~R.~G. 2010, Monthly Notices of the Royal Astronomical Society 
{\bf 405} 987

\bibitem[Brammer {\it et al}. 2008]{eazy}
Brammer,~G.~B., van Dokkum,~P.~G., and Coppi,~P. 2008,
The Astrophysical Journal {\bf 686}, 1503

\bibitem[Bryan 2007]{brentsthesis}
Bryan, B., 2007, Ph.D. thesis
\verb|http://reports-archive.adm.cs.cmu.edu/anon/|
\verb|ml2007/abstracts/07-122.html|

\bibitem[Bryan {\it et al}. 2007]{bryan}
Bryan, B., Schneider, J., Miller, C.~J., Nichol, R.~C., Genovese, C., and
Wasserman, L., 2007,
The Astrophysical Journal {\bf 665}, 25

\bibitem[Budav\'ari 2008]{budavari2008}
Budav\'ari,~T. 2008 The Astrophysical Journal {\bf 695}, 747

\bibitem[Collister and Lahav 2004]{annz}
Collister,~A.~A. and Lahav,~O. 2004,
Publications of the Astronomical Society of the Pacific {\bf 116}, 345

\bibitem[Connolly {\it et al}. 2005]{imsim}
Connolly,~A.~J., Peterson,~J., Jernigan,~J.~G., Abel,~R., Bankert,~J.,
Chang,~C., Claver,~C.~F., Gibson,~R., Gilmore,~D.~K., Grace,~E., Jones,~R.~L.,
Ivezic,~Z., Jee,~J., Juric,~M., Kahn,~S.~M., Krabbendam,~V.~L., Krughoff,~S.,
Lorenz,~S., Pizagno,~J., Rasmussen,~A., Todd,~N. Tyson,~J.~A., and Young,~M.
2005, Society of the Photo-Optical Instrumentation Engineers (SPIE) Converence
Series {\bf 7738}, 53

\bibitem[Cunha {\it et al}. 2012]{cunha2012}
Cunha,~C.~E., Huterer,~D., Lin,~H., Busha,~M.~T., and Wechsler,~R.~H. 2012,
[arXiv:1207.3347]

\bibitem[Daniel and Linder 2010]{muvarpi2}
Daniel,~S.~F. and Linder,~E.~V. 2010, Physical Review D {\bf 82}, 103523

%\cite{Daniel:2011rr}
\bibitem[Daniel {\it et al}. 2011]{daniel2011} 
  Daniel,~S.~F., Connolly,~A.~J., Schneider,~J., Vanderplas,~J. and Xiong,~L. 
  %``Classification of Stellar Spectra with LLE,''
  The Astronomical Journal  {\bf 142}, 203 (2011)
  [arXiv:1110.4646 [astro-ph.SR]].
  %%CITATION = ARXIV:1110.4646;%%

\bibitem[Daniel {\it et al}. 2012]{daniel2012}
Daniel,~S.~F., Connolly,~A.~J., and Schneider,~J. 2012
[arXiv:1205.2708]

\bibitem[Das {\it et al}. 2011]{sudeep}
Das,~S., de Putter,~R., Linder,~E.~V., and Nakajima,~R. 2011,
[arXiv:1102.5090]

\bibitem[Davis {\it et al}. 2007]{essence}
Davis,~T.~M., M\"ortsell,~E., Sollerman,~J., Becker,~A.~C., Blondin,~S.,
Challis,~P., Clocchiatti,~A., Filippenko,~A.~V., Foley,~R.~J., Garnavich,~P.~M.,
Jha,~S., Krisciunas,~K., Kirshner,~R.~P., Leibundgut,~B., Li,~W., Matheson,~T.,
Miknaitis,~G., Pignata,~G., Rest,~A., Riess,~A.~G., Schmidt,~B.~P.,
Smith,~R.~C., Spyromilio,~J., Stubbs,~C.~W., Suntzeff,~N.~B., Tonry,~J.~L.,
Wood-Vasey,~W.~M., and Zenteno,~A. 2007, The Astrophysical Journal, {\bf 666},
716

\bibitem[de Putter {\it et al}. 2010]{roland}
de Putter,~R., Huterer,~D. and Linder,~E.~V. 2010, Physical Review D {\bf 81},
103513


\bibitem[Djorgovski {\it et al}. 2011]{djorgovski2011}
Djorgovski,~S.~J., Donalek,~C., Mahabal,~A.~A., Moghaddam,~B., Turmon,~M.,
Graham,~M.~J., Drake,~A.~J., Sharma,~N. and Chen,~Y. 2011
[arXiv:1110.4655] to appear in Statistical Analysis and Data Mining, ref. proc.
CIDU 2011 conf., eds. A. Srivastava and N. Chawla

\bibitem[Garnett {\it et al}. 2011]{Garnett11}
Garnett,~R., Krishnamurhty,~Y., Wang,~D., Schneider,~J., and Mann,~R. 2011,
``Bayesian Optimal Active Search on Graphs,'' KDD Workshop on Mining and
Learning with Graphs

\bibitem[Garnett {\it et al}. 2012a]{Garnett12}
Garnett,~R., Krishnamurthy,~Y., Xiong,~X., Schneider,~J., and Mann,~R. 2012a,
``Bayesian Optimal Active Search and Surveying,'' International Conference on
Machine Learning

\bibitem[Garnett {\it et al}. 2012b]{Garnett12a}
Garnett,~R., Ho,~S., and Schneider,~J. 2012b,
``Gaussian Processes for Identifying Damped Lyman-alpha Systems in Spectroscopic
Surveys,'' Neural Information Processing Systems 
workshop on Modern Nonparametric Methods in Machine Learning

\bibitem[Graham {\it et al}. 2012]{graham2012}
Graham,~M.~J., Djorgovski,~S.~G., Mahabal,~A., Donalek,~C., Drake,~A.,
Longo,~G. 2012 [arXiv:1208.2480] to appear in special issue of Distributed and
Parallel Databases on Data Intensive eScience

\bibitem[Huterer {\it et al}. 2006]{huterer2006}
Huterer,~D., Takada,~M., Bernstein,~G., and Jain,~B. 2006,
Monthly Notices of the Royal Astronomical Society {\bf 366}, 101

\bibitem[Kitching {\it et al}. 2008]{kitching}
Kitching,~T.~D., Taylor,~A.~N., and Heavens,~A.~F. 2008,
Monthly Notices of the Royal Astronomical Society {\bf 389} 173

\bibitem[Long {\it et al}. 2012]{long2012}
Long,~J.~P., El Karoui,~N., Rice,~J.~A., Richards,~J.~W., and Bloom,~J.~S. 2012,
Publications of the Astronomical Society of the Pacific {\bf 124} 280

\bibitem[LSST Collaboration 2011]{lsstoverview}
LSST Collaboration 2011, [arXiv:0805.2366]
\verb|http://www.lsst.org/lsst/overview/|

\bibitem[LSST Dark Energy Science Collaboration 2012]{desc}
LSST Dark Energy Science Collaboration 2012, [arXiv:1211.0310]

\bibitem[LSST Science Collaborations 2009]{sciencebook}
LSST Science Collaborations 2009, ``LSST Science Book'',
\verb|http://www.lsst.org/lsst/science/scibook|

\bibitem[Ma {\it et al}. 2006]{Ma2006}
Ma,~Z., Hu,~H., and Huterer,~D. 2006, The Astrophysical Journal {\bf 636}, 21

\bibitem[Ma {\it et al}. 2012]{YifeiMa12}
Ma,~Y., Garnett,~R., and Schneider,~J. 2012,
``Submodularity in Batch Active Learning and Survey Problems
on Gaussian Random Fields,''
Neural Information Processing Systems 
workshop on Discrete Optimization in Machine Learning

\bibitem[Mahabal {\it et al}. 2008a]{mahabal2008a}
Mahabal,~A., Djorgovski,~S.~G., Turmon,~M., Jewell,~J., Williams,~R.~R.,
Drake,~A.~J., Graham,~M.~G., Donalek,~C., Glikman,~E., and the Palomar-QUEST Team
2008a, Astronomische Nachrichten {\bf 329}, 288

\bibitem[Mahabal {\it et al}. 2008b]{mahabal2008b}
Mahabal,~A., Djorgovski,~S.~G., Williams,~R., Drake,~A., Donalek,~C.,
Graham,~M., Moghaddam,~B., Turmon,~M., Jewell,~J., Khosla,~A., and
Hensley,~B. 2008b [arXiv:0810.4527] to appear in proceedings fo the Class2008
conference (Classification and Discovery in Large Astronomical Surveys, Ringberg
Castle, 14-17 October 2008)

\bibitem[Mahabal {\it et al}. 2011a]{mahabal2011a}
Mahabal,~A.~A., Donalek,~C., Djorgovski,~S.~J., Drake,~A.~J.,
Graham,~M.~J., Williams,~R., Chen,~Y., Moghaddam,~B., and Turmon,~M.
2011a, [arxiv:1111.3699] to appear in Proc. IAU 285, ``New Horizons in Transient
Astronomy,'' Oxford, September 2011

\bibitem[Mahabal {\it et al}. 2011b]{mahabal2011b}
Mahabal,~A.~A., Djorgovski,~S.~G., Drake,~A.~J., Donalek~C., Graham,~M~J.,
Williams,~R.~D., Chen,~Y., Moghaddam,~B., Turmon,~M., Beshore,~E., and Larson,~S.
2011b, Bulletin of the Astronomical Society of India {\bf 39}, 387

\bibitem[Mandelbaum {\it et al}. 2008]{mandelbaum2008}
Mandelbaum,~R., Seljak,~U., Hirata,~C.~M., Bardelli,~S., Bolzonella,!M.,
Bongiorno,~A., Carollo,~M., Contini,~T., Cunha,~C.~E., Garilli,~B.,
Iovino,~A., Kambczyk,~P, Kneib,~J.-P., Knobel,~C., Koo,~D.~C., Lamareille,~F.,
Le F\`evre,~O., Leborgne,~J.-F., Lilly,~S.~J., Maier,~C., Mainieri,~V.,
Mignoli,~M., Newman,~J.~A., Oesch,~P.~A., Perez-Montero,~E., Ricciardelli,~E.,
Scodeggio,~M., Silverman,~J., and Tasca,~L. 2008, Monthly Notices of the Royal
Astronomical Society {\bf 386}, 781

\bibitem[McBride {\it et al}. 2011a]{mcbride2011a}
McBride,~C.~K., Connolly,~A.~J., Gardner,~J.~P., Scranton,~R., Newman,~J.~A.,
Scoccimarro,~R., Zehavi,~I., and Schneider,~D.~P. 2011a, The Astrophysical
Journal, {\bf 726}, 13

\bibitem[McBride {\it et al}. 2011b]{mcbride2011b}
McBride,~C.~K., Connolly,~A.~J., Gardner,~J.~P., Scranton,~R., Scoccimarro,~R.,
Berlind,~A.~A., Mar\'in,~F., and Schneider,~D.~P. 2011b, The Astrophysical
Journal {\bf 739}, 85

\bibitem[Moore {\it et al}. 2000]{Moore00}
Moore,~A., Connolly,~A., Genovese,~C., Grone,~L., Kanidoris,~N., Nichol,~R.,
Schneider,~J., Szalay,~A., Szapudi,~I., and Wasserman,~L. 2000,
``Fast Algorithms and Efficient Statistics: N-point Correlation Functions,'' in
MPA/MPE/ESO Conference on Mining the Sky [arXiv:astro-ph/0012333]

\bibitem[Nakajima {\it et al}. 2012]{nakajima2011}
Nakajima,~R., Mandelbaum,~R., Seljak,~U., Cohn,~J.~D., Reyes,~R., and
Cool,~R. 2012, Monthly Notices of the Royal Astronomical Society {\bf 420}, 3240
[arXiv:1107.1395]

\bibitem[Nichol {\it et al}. 2006]{Nichol2006}
Nichol,~R.~C., Sheth,~R.~K., Suto,~Y., Gray,~A.~J., Kayo,~I., Wechsler,~R.~H.,
Marin,~F., Kulkarni,~G., Blanton,~M., Connolly,~A.~J., Gardner,~J.~P., Jain,~B.,
Miller,~C.~J., Moore,~A.~W., Pope,~A., Pun,~J., Schneider,~D., Schneider,~J.,
Szalay,~A., Szapudi,~I., Zehavi,~I., Bahcall,~N.~A., Csabai,~I., Brinkmann,~J.
2006, Monthly Notices of the Royal Astronomical Society {\bf 368}, 1507

\bibitem[Poczos and Schneider 2011]{poczos11alphadiv}
Poczos,~B. and Schneider,~J. 2011, ``On the Estimation of alpha-Divergences,''
Artificial Intelligence and Statistics (AISTATS)

\bibitem[Poczos {\it et al}. 2011]{Poczos2011UAI}
Poczos,~B., Xiong,~L., and Schneider,~J. 2011, ``Nonparametric Divergence Estimation with
Applications to Machine Learning on Distributions,''  Uncertainty in Artificial
Intelligence

\bibitem[Poczos {\it et al}. 2012]{poczos12CVPR}
Poczos,~B., Xiong,~L., Sutherland,~D., and Schneider,~J. 2012,
``Nonparametric Kernel Estimators for Image Classification,''
IEEE Conference on Computer Vision and Pattern Recognition

\bibitem[Rasmussen and Williams 2006]{gp}
Rasmussen, C.~E. and Williams, C.~K.~I., 2006, ``Gaussian
Processes for Machine Learning''
\verb|http://www.GaussianProcess.org/gpml/|

\bibitem[Richards {\it et al}. 2004]{qso}
Richards,~G.~T., Nichols,~R.~C., Gray,~A.~G., Brunner,~R.~J., Lupton,~R.~H.,
Vanden Berk,~D.~E., Chong,~S.~S., Weinstein,~M.~A., Schneider,~D.~P.,
Anderson,~S.~F., Munn,~J.~A., Harris,~H.~C., Strauss,~M.~A., Fan,~X.,
Gunn,~J.~E., Ivezi\'c,~Z., York,~D.~G., Brinkmann,~J., and Moore,~A.~W. 2004,
The Astrophysical Journal Supplement Series, {\bf 155}, 257

\bibitem[Richards {\it et al}. 2011]{richards2011}
Richards,~J.~W., Starr,~D.~L., Butler,~N.~R., Bloom,~J.~S., Brewer,~J.~M.,
Crellin-Quick,~A., Higgins,~J., Kennedy,~R., and Rischard,~M. 2011,
The Astrophysical Journal {\bf 733}, 10

\bibitem[Richards {\it et al}. 2012]{richards2012}
Richards,~J.~W., Starr,~D.~L., Brink,~H., Miller,~A.~A., Bloom,~J.~S.,
Butler,~N.~R., James,~J.~B., Long,~J.~P., and Rice,~J. 2012
The Astrophysical Journal {\bf 774}, 192

\bibitem[Rosenfield {\it et al}. 2011]{rosenfield2011}
Rosenfield,~P., Connolly,~A., Fay,~J., Sayres,~C., and Tofflemire,~B. 2011,
Astronomical Society of the Pacific Conference Series {\bf 443}, 109

\bibitem[Scranton {\it et al}. 2002]{Scranton2002}
Scranton,~R., Johnston,~D., Dodelson,~S., Frieman,~J.~A., Connolly,~A.,
Eisenstein,~D.~J., Gunn,~J.~E., Hui,~L., Jain,~B., Kent,~S., Loveday,~J.,
Narayanan,~V., Nichol,~R.~C., O'Connell,~L., Soccimarro,~R., Sheth,~R.~K.,
Stebbins,~A., Strauss,~M.~A., Szalay,~A.~S., Sapudi,~I., Tegmark,~M.,
Vogeley,~M., Zehavi,~I., Annis,~J., Bahcall,~N.~A., Brinkman,~J., Csabai,~I.,
Hindsley,~R., Ivezic,~Z., Kim,~R.~S.~J., Knapp,~G.~R., Lamb,~D.~Q., Lee,~B.~C.,
Lupton,~R.~H., McKay,~T., Munn,~J., Peoples,~J., Pier,~J., Richards,~G.~T.,
Rockosi,~C., Schlegel,~D., Schneider,~D.~P., Stoughton,~C., Tucker,~D.~L.,
Yanny,~B., York,~D.~G. 2002, The Astrophysical Journal {\bf 579}, 48

\bibitem[Sesar {\it et al}. 2011]{linear}
Sesar,~B., Stuart,~J.~S., Ivezi\'c,~\u Z., Morgan,~D.~P., Becker,~A.~C., and
Wo\'zniak,~P. 2011, The Astronomical Journal {\bf 142}, 190

\bibitem[Settles 2009]{activelearning}
Settles,~B. 2009, ``Active Learning Literature Survey,'' Computer Sciences Technical
Report 1648, University of Wisconsin-Madison,
\verb|http://pages.cs.wisc.edu/~bsettles/active-learning/|


\bibitem[Shafieloo {\it et al}. 2012]{ericgp}
Shafieloo,~A., Kim,~A.~G., and Linder,~E.~V. 2012,
Physical Review D {\bf 85}, 123530 [arXiv:1204.2272]


\bibitem[Skibba {\it et al}. 2006]{Skibba2006}
Skibba,~R., Sheth,~R.~K., Connolly,~A.~J., and Scranton,~R. 2006,
Monthly Notices of the Royal Astronomical Society, {\bf 369}, 68

\bibitem[Straf 2003]{straf03}
Straf,~M.~L. 2003, Journal of the American Statistical Association {\bf 98}, 1

\bibitem[Szapudi {\it et al}. 2002]{Szapudi2002}
Szapud,~I., Frieman,~J.~A., Scoccimarro,~R., Szalay,~A.~S., Connolly,~A.~J.,
Dodelson,~S., Eisenstein,~D.~J., Gunn,~J.~E., Johnston,~D., Kent,~S.,
Loveday,~J., Meiksin,~A., Nichol,~R.~C., Scranton,~R., Stebbins,~A.,
Vogeley,~M.~S., Annis,~J., Bahcall,~N.~A., Brinkman,~J., Csabai,~I., Doi,~M.,
Fukigita,~M., Ivezi\'c,~\u Z., Kim,~R.~S.~J., Knapp,~G.~R., Lamb,~D.~Q.,
Lee,~B.~C., Lupton,~R.~H., McKay,~T.~A., Munn,~J., Peoples,~J., Pier,~J.,
Rockosi,~C., Schlegel,~D., Stoughtfon,~C., Tucker,~D.~L., Yanny,~B., York,~D.~G.
2002, The Astrophysical Journal {\bf 570}, 75

\bibitem[Vanderplas and Connolly 2009]{vdp2009}
Vanderplas,~J. and Connolly,~A.~J. 2009,
Astronomical Journal {\bf 138}, 1365

\bibitem[Wiley {\it et al}. 2011]{wiley2011}
Wiley,~K, Connolly,~A.~J., Gardner,~J., Krughoff,~S., Balazinska,~M., Howe,~B.,
Kwon,~Y., and Bu, ~Y. 2011, Publication of the Astronomical Society of the
Pacific {\bf 123}, 366

\bibitem[Xiong {\it et al}. 2011a]{Xiong2011gad}
Xiong,~L., Poczos,~B., Schneider,~J., Connolly,~A., Vanderplas,~J. 2011a,
``Hierarchical Probabilistic Models for Group Anomaly Detection,''
Artificial Intelligence and Statistics (AISTATS)

\bibitem[Xiong {\it et al}. 2011b]{xiong2011fgm}
Xiong,~L., Poczos,~B., and Schneider,~J. 2011, ``Group Anomaly Detection using Flexible
Genre Models,'' Neural Information Processing Systems

\bibitem[Yip {\it et al}. 2004]{yip2004a}
Yip,~C.~W., Connolly,~A.~J., Szalay,~A.~S., Budav\'ari,~T., SubbaRao,~M.,
Frieman,~J.~A., Nichol,~R.~C., Hopkins,~A.~M., York,~D.~G., Okamura,~S.,
Brinkmann,~J., Csabai,~I., Thakar,~A.~R., Fukugita,~M., 
and Ivezi\'c,~\u Z. 2004, The Astronomical Journal {\bf 128}, 585

\bibitem[Zhang and Schneider 2010a]{YiZhangICML2010}
Zhang,~Y. and Schneider,~J. 2010a, ``Projection Penalties: Dimension Reduction without
Loss,'' International Conference on Machine Learning

\bibitem[Zhang and Schneider 2010b]{YiZhangMultitask2010}
Zhang,~Y. and Schneider,~J. 2010b,
``Learning Multiple Tasks with a Sparse Matrix-Normal Penalty,''
Neural Information Processing Systems

\bibitem[Zhang {\it et al}. 2010]{YiZhangSDM2010}
Zhang,~Y., Schneider,~J., and Dubrawski,~A. 2010,
``Learning Compressible Models,'' Proceedings of SIAM Data Mining Conference

\bibitem[Zhang and Schneider 2011]{YiZhang2011multiECOC}
Zhang,~Y. and Schneider,~J 2011, ``Multi-label Output Codes using Canonical Correlation
Analysis,'' Artificial Intelligence and Statistics

\bibitem[Zhang and Schneider 2012]{YiZhang2012}
Zhang,~Y. and Schneider,~J. 2012, ``Maximum Margin Output Coding,''
International Conference on Machine Learning

\end{thebibliography} 
\label{lastpage}

\end{document}

%%%%%%%%%%%%%%%%%%%%%%%%%%%%%%%%%%%%%%%%%%%%%%%%%%%%%%%%%%%%
