\documentclass[useAMS,usenatbib,tightenlines,11pt,preprint]{aastex}
%\documentclass[useAMS,usenatbib,tightenlines,11pt,preprint]{aastex}
\usepackage[paperwidth=8.5in,paperheight=11in,centering,margin=1in]{geometry}

\usepackage{parskip}
%\setlength{\parskip}{\baselineskip}
\parskip=5pt

\usepackage{amsmath}
\usepackage{amsbsy}

\input epsf
\usepackage{amsmath,amssymb,subfigure}
\usepackage{graphicx}
\usepackage{epsfig}
\usepackage{color}
%\usepackage{ulem}
%\usepackage{epstopdf}

\usepackage{multicol}
%\usepackage{etoolbox}

\pagestyle{empty}

\renewcommand{\baselinestretch}{0.99}

%%%%%%%%%%%%%%%%%%%%%%%%%%%%%%%%%%%%%%%%%%%%%%%%%%%%%%%%%%%%
%%%%%%%%%%%%%%%%%%%%%%%%%%%%%%%%%%%%%%%%%%%%%%%%%%%%%%%%%%%%
%%%%%%%%%%%%%%%%%%%%%%%%%%%%%%%%%%%%%%%%%%%%%%%%%%%%%%%%%%%%

\begin{document} 

\begin{center}
{\bf \Large Learning in an Era of Uncertainty}
\end{center}

\vspace{1cm}

\noindent
\begin{tabular}{ll}
{\bf Principal Investigator: } &Andrew Connolly \\
{\bf Applicant/institution: } & University of Washington\\
{\bf Telephone number: } & (206) 543 9541 \\
{\bf Email: } & ajc@astro.washington.edu \\
{\bf Funding request:} & \$170K (year one), \$170K (year two), \$170K (year
three) \\
{\bf DOE/OSP Office: } & Office of Advanced Scientific Computing Research \\
{\bf DOE/Technical Contact: } & Dr. Alexandra Landsberg 
\end{tabular}

\vspace{1cm}


\noindent \begin{tabular}{ll}
{\bf CoPrincipal Investigator: } & Jeff Schneider \\
{\bf Applicant/institution: } & Carnegie Mellon University \\
{\bf Telephone number: } & (412) 268 2339 \\
{\bf Email: } & schneide@cs.cmu.edu \\
{\bf Funding request:} & \$170K (year one), \$170K (year two), \$170K (year 
three) 
\end{tabular}
 
\vspace{1cm}

\noindent \begin{tabular}{ll}
{\bf Total Funding request:} & \$340K (year one), \$340K (year two), \$340K
(year  three)
\end{tabular}

% \label{firstpage}

% \maketitle 
\newpage

\begin{center}
{\bf \Large Learning in an Era of Uncertainty}
\end{center}

\section{Introduction:}
A new generation of DOE sponsored data intensive experiments and
surveys, designed to address fundamental questions in physics and
biology, will come on-line over the next decade. These experiments
share many similar challenges in the field of statistics and
machine-learning: how can we choose the next experiment or observation
to make in order that we maximize our scientific returns; how can we
identify anomalous sources (that may be indicative of new events or
potential systematics within our experiments) from a continuous stream
of data; how do we characterize and classify correlations and events
within data streams that are noisy and incomplete. The goal of this
proposal is to address these challenges through the development of a broad
class of novel and scalable machine-learning techniques centered
around the theme of active learning.

{\it Active learning} algorithms iteratively decide on which data
points they will collect outputs on and add to a training set.  Their
goal is to choose the points that will most improve the model being
learned.  At each step, they consider the current training data, the
potential training data that could be collected, and the current
learned model, and evaluate what would be the best choice for the next
observation, experiment, or feature such that it improve the our
knowledge of the system (according to some objective criterion).
% something more on their use and potential

% While the basic formalism for active learning is well understood
% there are many challenges associated with data streams...

While the algorithms and methodologies we propose will impact many of
the data intensive sciences, we will focus our work in the context of
DOE sponsored cosmology experiments (i.e.\ the Dark Energy Survey, and
the Large Synoptic Survey Telescope). These surveys are ideal proxies
as their bandwidth (terabytes of data per night and petabytes of data
every couple of months) will enable high precision studies impacting
the understanding of cosmology, particle physics, and potentially
theories of gravity.  Our ability to achieve these scientific goals
relies on analyses at a scale, speed, and complexity beyond the
capabilities of current automated machine learning methods.


\section{The Need for Active Learning in Cosmology}

Over the last decade a concordance model has emerged for the universe
that describes its energy content. The most significant contribution
to the energy budget today comes in the form of ``dark energy'', which
explains the observation that we reside in an accelerating
universe. Despite its importance to the formation and evolution of the
universe there is no compelling theory that explains the energy
density nor the properties of the dark energy. Understanding the
nature of dark energy remains as one of the most fundamental questions
in Physics today, impacting our understanding of cosmology, particle
physics, and potentially theories of gravity itself.  As noted in the draft
report of the Dark Energy Task Force (DETF; constituted jointly by
DOE, NSF and NASA), ``the nature of dark energy ranks among the very
most compelling of all outstanding problems in physical science''.


To address the question of the nature of dark energy a new generation
of DOE sponsored experiments are entering service (e.g.\ the Dark
Energy Survey, DES\footnote{http://www.darkenergysurvey.org} and the
Large Synoptic Sky Survey, LSST\footnote{http://www.lsst.org}).  These
surveys will represent a 40-fold increase in data rates over current
experiments (generating over 100 Petabytes of data over a period of 10
years) and decreasing the uncertainties on our measures of the
underlying properties of dark energy by more than a factor of ten. On
these scales, statistical noise will no longer determine the accuracy
to which we can measure cosmological parameters. The control and
correction of systematics will ultimately determine our final
figure-of-merit. Prime amongst these systematics is the estimation of
distances to extragalactic sources, the identification of anomalous
events within the temporal universe (e.g.\ detecting optical flashes
and supernovae at cosmological distances), and the real time
classification of data in the presence of uncertainties and gaps
within the data stream.

We propose to address these problems by using Gaussian Processes to model the
probability distributions underlying the data.  We believe that this framework
is the most robust and informative available.  Gaussian Processes have already
demonstrated their strength at inferring the form of the latent functions
underlying data \cite{ericgp,psf,daniel2012}, even in cases of sparse
measurements.  By using them to model the
probabilistic nature of the data, we will harness that power to learn, not only
the models underlying the data, but the uncertainties surrounding that model,
and the potential information to be gained from different follow-up
observations.  These three inferences will be critical in an age of
research with low tolerance for uncertainty and limited budgets for
follow-up observation. 

%the identification and characterization of novel
%sources requires sifting through millions of ``uninteresting'' objects
%to locate those truly worth studying.  The next generation of
%astronomy will be defined by large surveys such as
%LSST\footnote{http://www.lsst.org}, each of which promises to yield
%terabytes of data every night.  

%These surveys will pose challenges not
%only because of the volume of data they will provide, but the type, as
%their repeated observations of the same regions of sky will finally
%open detailed time-domain astronomy to physical scrutiny
%\cite{sciencebook}.  Given the billion dollar investments being made
%in these surveys, it is paramount that we develop techniques that can%
%optimize their science output.

\section{Probabilistic Framework for Scientific Inference}

\subsection{Gaussian Processes}
\label{sec:gppz}

Gaussian processes (GPs) model the output of an unknown, noisy,
sparsely-measured
function as though it was generated by a random process with an underlying
Gaussian probability distribution \cite{gp}.
An important feature of GPs is that they do not make parametric assumptions
about the form of the function they are modeling.  They have received
attention as a means of describing physical phenomena (e.g. the expansion
history of the universe) without having to assume a model of the underlying
process \cite{ericgp}, of interpolating point spread functions across large
images \cite{psf}, and of accelerating the search of high-dimensional
likelihood functions on the cosmological parameters by efficiently
selecting sample points \cite{daniel2012}.  

Assume that each training set datum is of the form
$\{\vec{\theta},y\}$, where $\vec{\theta}$ is an $N_p$-dimensional vector
respresenting the measured data and $y = f(\theta)$ is
the latent quantity we are trying to infer.  Gaussian
processes assume that $f$ is a probabilistic function on the
$N_p$-dimensional space with some covariance function such as a squared
exponential covariance,
$K_{ij}\equiv\text{Cov}\left[f(\vec{\theta}^{i}),f(\vec{\theta}^{j})\right]
= \exp(-\frac{1}{2}|\vec{\theta}^{i} - \vec{\theta}^{j}|^2)$.
Under those assumptions, one can derive a posterior probability distribution
for the redshift at a new query point $\{\vec{\theta}_{q}\}$ by marginalizing
over the measured points $\{\vec{\theta}\}$.  
This gives a Gaussian distribution with mean 

\begin{equation}
f(\vec{\theta}_{q}) = \sum_{i=1}^n \
\alpha_i k(\vec{\theta}^{i},\vec{\theta}_{q}^{i})
\label{eq:mean}
\end{equation}

\noindent
and variance

\begin{equation}
\Sigma_{q} = \text{Cov}(f_{q}) = K_{qq} - K_q^T (K + \sigma^2I)^{-1} K_q
\label{eq:cov}
\end{equation}

\noindent
Here, $\vec{\alpha} = (K + \sigma^2 I)^{-1}\vec{y}$, $K$ is the matrix of
covariances between all training points, $\sigma^2$ is the variance of
Gaussian noise added to each observed value of $y$, $\vec{y}$
represents the training data redshifts, and
where $K_q$ is the vector of covariances between the query point and all
training points.  Eq. \ref{eq:cov} extends directly to the case of multiple
query points by making the variables as matrices where appropriate, and the
result is a full covariance matrix for the query points.  Readers looking
for a more detailed explanation of Gaussian processes should consult
Rasmussen and Williams (2006).

Gaussian Processes have already received significant attention in the field of
machine learning, but primarily for their felicitous qualities as function
regressors, rather than for their ability to directly model the probability
distribution of data.  Wang {\it et al}. (2011, 2012) use a mixture of Gaussian
Processes to model light curves and identify periodically varying stars.  Huisje
{\it et al}. (2011, 2012) improve upon their methods, using information
theoretic quantities to separate true period of these variations from
systematics introduced by observing systems.  We will extend this work
to develop an algorithm that returns a full distribution of probabilities
$P(f^q)$ across all possible values (whether a continuous function or a
classification) for $f^q$.

Furthermore, we will use the probabilistic nature of Gaussian Processes 
(specifically, equation \ref{eq:cov}) to provide a direct
quantification of the information content of the model.  This will
allow us to
identify the least-well constrained test datum for follow-up observations.  It
will also allow 
us to project the impact of such follow-up on the model of the entire 
data set \cite{Garnett11,Garnett12}.


\subsection{Inference in the Presence of Noise and Gaps in a Data
  Stream}

The non-parametric inferences produced by
Gaussian Processes yield smooth functions which are sensitive to the
correlations in the input data.  Because of this, Gaussian Processes are
often used to solve regression problems in the case of
sparsely sampled training data \cite{wang2011,wang2012}.  
We intend to harness these features to produce classification algorithms that
are resilient against noise and gaps in our training data.

Figure \ref{fig:desert} compares a Gaussian Process classifier (the
photometric redshift algorithm described in Section \ref{sec:photoz}) to 
an artificial neural network when both are trained on
data whose distribution does not match the true distribution of data.  
The classification is a scalar function $f$.  
The data points $\{\vec{\theta}\}$
exist in a 6-dimensional space.  
We treat the algorithms as returning a probability distribution over $f$.  The horizontal axis is
the mode of these returned probability distributions (effectively, $f_q$ from equation
\ref{eq:mean}).  The vertical axis is the value of the probability distribution returned at the
true value of $f$.
The thin lines are
produced when all of the training data with values $1.4\le f\le
2.5$ (between the vertical green lines).  In this case, the artificial neural network
(red curves) does demonstrably better than the Gaussian Process classifier (black curves).
The thick lines are produced when only 5\% of the excised training data is restored.
Now we see that the Gaussian Process classifier learns much faster and does a much better job
of classifying data in the presence of this sparse training distribution.

This is a very basic test.  The Gaussian Process was not allowed to choose optimal observations or
training data, nor was much thought given to selecting the appropriate cost function used when
designing the Gaussian Process model.  Incorporation of active learning based on the information
content encoded in the probability distributions returned by our Gaussian Process algorithm will
yield further gains in performance over current classifiers (e.g. the artificial 
neural network considered below).

\begin{figure}
\centerline{\includegraphics[scale=0.3]{doe_desert_plot.png}}
\caption{
We compare the Gaussian Process algorithm outlined in Section \ref{sec:photoz}
to an artificial neural network in the case of sparse training data.  The red
curves represent an artificial neural network scheme.  The black curves
represent the Gaussian Process algorithm presented in Section \ref{sec:photoz}. 
The thin curves represent the case where all of the training data between the
dashed lines has been excised.  The thick curves represent the case where 5\% of
that training data has been restored.
}
\label{fig:desert}
\end{figure}

\subsection{Anomaly Detection and Classification in Massive Data
 Streams}

The next generation of astrophysical surveys we will visit the same
region of sky many thousands of times. This opening of the temporal
domain in astrophysics offers the potential to discover new classes of
physical phenomena while coming with many associated computational
challenges. Variability within the universe is believed to be present
on time scales of seconds through to tens of years. The shortest time
scales correspond to the explosion of the most massive stars within
the universe which produce short but intense optical and gamma-ray
flashes. These outbursts provide direct tests of General Relativity
and of high energy physical processes (at energies far beyond those
accessible on the Earth). For example the rate at which these events
occur constrains the age at which the first stars within the universe
came into being. Intermediate timescale variability comes in the form
of supernovae (SNe) which detonate, brighten and then dim. These
exploding stars are known to have a narrow range of intrinsic
brightnesses; they act as standard candles that can be used to
determine the rate at which the universe expands and thereby measure
its mass and energy content \cite{perlmutter99}. 
%Longer time scale
%variablity (including variability due to the velocity of sources)
%comes from variations in the luminosity of acretion disks around black
%holes, the motion of stars throughout our Galaxy and the motion of
%asteroids within the local solar system. These...
With surveys such as the LSST we will detect 250,000 SNe per year
increasing the accuracy of measures of the energy content of the
universe by an order of magnitude.

With timescales as short as seconds to hours we need to be able to
identify, classify and report any detection in time to allow for
follow-up observations before the initial outburst
fades. Identification and classification must, therefore, be
undertaken in almost real-time with probabilistic classifications that
incorporate our uncertainties about our classification together with
the ability for algorithms to learn based on a posteriori information
from earlier classifications. It must be able to predict what
additional information might be needed to improved (or exclude) the
likelihood of a given classification and to specify which parameters
led to the source being classified as anomalous. Small errors in the
identification and classification of these anomalous sources will
swamp any underlying signal. The LSST will detect 7.5x10$^8$ sources
{\bf every night}. Even for the most numerous transient events (SNe)
this corresponds to less than 10$^{-5}$ of the total number of sources
identified being transient. For the most energetic bursters the
magnitude of the challenge is 500-fold larger.  Algorithms for
identifying anomalies and variability must, therefore, be robust to
false positives and missing data and must account for the cadence in
how we sample the time domain, variations in the quality of the data
due to atmospheric conditions, changes in the performance of the
telescope and camera and the possibility that we observe sources at
different wavelengths at different times.



\begin{figure}
\centerline{
\subfigure[]{
\includegraphics[scale=1.0]{becker_rrlyrae_uc.png}
\label{subfig:uc}
}
\subfigure[]{
\includegraphics[scale=1.0]{becker_rrlyrae_dd.png}
\label{subfig:dd}
}
}
\centerline{
\subfigure[]{
\includegraphics[scale=1.0]{becker_rrlyrae_year_1.png}
\label{subfig:year1}
}
\subfigure[]{
\includegraphics[scale=1.0]{becker_rrlyrae_year_5.png}
\label{subfig:year5}
}
}
\caption{
Taken from \cite{rrlyrae}.  Figure \ref{subfig:uc} shows the sampling of
a template RR Lyrae light curve at the LSST Universal Cadence.
Figure \ref{subfig:dd} shows the same for the 
Deep Drilling Cadence.  Figures \ref{subfig:year1} and \ref{subfig:year5} compare the scatter
in determining the phase of an RR Lyrae light curve after 1 year of Universal Cadence
data and 5 years of Universal Cadence data.  This illustrates the significant effect
on scientific output of more and more targeted data.
}
\label{fig:RRLyrae}
\end{figure}





\section{Experimental Design and Optimization through Active
  Learning}
 
{\bf Experimental design description}


A classic active learning method, called uncertainty sampling, uses
the uncertainty of each test point as the criterion for choosing the
next experiment (e.g.\ the next spectroscopic measurement of a source
in the training sample).  
Figure \ref{fig:learning} demonstrates a variant of this scheme assigned to the
problem presentedin Figure \ref{fig:desert}. From a total of 97,000 data points,
we start with 20,000 training points and assess the efficacy of our Gaussian
Process classifier by considering the mean value of the probability distribution
returned at the true value of the classifying function $f$.  The training set is
then grown by selecting new points either randomly (black curve) or according to
the maximum value of $\left(-\ln[P(\text{mode})]\right)$ (red curve).  As you can see,
assembling the training set with active learning leads to a significant
improvement in the classifier's performance.
We will expand upon these approaches using
our recent work on the problem of optimal surveying or polling
(Garnett et al 2012a).  Rather than having a goal of correctly
predicting the output for each point in a test set, the goal is to
predict the average output (or the class proportions in classification
problems) over the test set.  This dramatically increases the
efficiency of the active learning.  In preliminary experiments on
graphs and other domains, minimizing this survey variance not only
performs well on the surveying problem, but also outperforms the trace
criterion and other popular active learning methods such as
uncertainty and density sampling on active learning problems.
This result is consistent with the findings of Richards {\it et al.} (2012b),
who consider a similar problem for a Random Forest classifier on 
the static data set produced by the All Sky Automated Survey.
Intuitively, it seems reasonable that considering the entire
covariance matrix might lead to better performance than choosing only
based on its diagonal.  We have, however, little theoretical understanding of
why this is better than the trace criterion which directly optimizes
the quantity on which we will ultimately measure performance.  We will
seek a better theoretical understanding of this phenomenon as part of
this work.

\begin{figure}
\centerline{
\includegraphics[scale=0.3]{learning_curve.png}
}
\caption{
Active learning applied to classification according to a scalar function on a
6-dimensional data space.  The horizontal axis is the size of the training data
set.  The vertical axis is the mean value of the output probability distribution
at the true value of the scalar function.  The black curve assembles the
training set randomly.  The red curve selects new training points that maximize
the figure of merit $\left(-\ln[P(\text{mode})]\right)$.
}
\label{fig:learning}
\end{figure}

\subsection{Active Learning for Transient Classification}


{\bf I have just copied-and-pasted the ``active learning for transients''
paragraphs from the previous draft}


Real-time automatic classification of objects is already widely acknowledged as a
necessary support technology for the forthcoming age of survey astronomy
\cite{djorgovski2011,richards2011,richards2012,graham2012,mahabal2008a,mahabal2011a}.
Objects will need to be categorized into known science classes so that novel
or rare objects can be flagged for detailed follow-up observations.
For transient events, algorithms must be able to make rapid
decisions so that sources can be targeted for follow-up 
and classifications learned before
objects return to their quiescent phases.  

A great deal of work has
already been done developing algorithms that can learn the classification of an
object given a fixed set of observations and training data.  
Mahabal {\it et al}. (2008a,2011a,2011b) propose to break down the
observations of a given object into $\{\Delta m,\Delta t\}$ pairs (where $m$
is magnitude and $t$ is time) and use the density of observations in this 
two-dimensional space as the basis for a Bayesian classification algorithm.
Mahabal {\it et al}. (2008b) alternatively propose to use those same 
$\{\Delta m,\Delta t\}$ pairs as the input to a Gaussian process regression
by which they will reconstruct the object's entire light curve as a function
of time, and then classify the object based on that reconstruction.
Anomaly detection has been attempted by decomposing light curves into
basis functions of different timescales and looking for events that occur
more rapidly than some fit baseline \cite{preston2009,blocker2013}.
Richards {\it et al}. (2011) use observations of transient objects to extract
periodic (e.g. the amplitude and frequency of the first two Fourier modes of
the object's light curve) and non-periodic (e.g. the variance and skewness of
all of the magnitude observations taken, regardless of their separation in
time) and feed those features into several tree-based classifiers.
They find misclassification rates lower than 30\% with their best method yielding a
misclassification rate of 22.8\%.  Using only non-periodic features, which will be
especially easy for survey telescopes to gather, rather than
full light curves, they 
find a misclassification rate of
between 26\% and 28\%.
Bloom {\it et al}. (2011) also use a tree-based automatic classifier on
Palomar Transient Factory data and find a 3.8\% error rate when
discriminating between four major classifications.  Richards {\it et al}.
consider a more complete set of 25 possible classifications.
Clearly, many possibile approaches are available for the automated
classification of transient objects, and not all of them rely upon highly
detailed observations to function.  
None of the above algorithms, however, make
any promises regarding their ability to deliver rapid recommendations for
optimum follow-up observations in real time.  This will be a significant
contribution to all time-sensitive sciences.

Similarly, a large number of classifiers have been developed which are optimized
for the case of binary classifications: ``Is something a quasar, or is it not?''
\cite{kim2011,pichara2012}, ``Is an object at redshift greater than 4 or is it
not?'' \cite{morgan2011}, ``Is an object a real astrophysical transient, or is
it an instrumental artifact?'' \cite{brink2012}.  The methods developed
(Random Forests; Support Vector Machiens; etc.) are all useful
and provide guidance for what can and should be attempted, however, they do not
deliver the robust, probabilistic, multi-class identifiers that will be required
by the rapid, big-data experiments of the next decades.

\subsection{Proposed method for active learning classifier}

The input features for transient object classification will be photometric
and morphologic measures taken at different increments in time.  The output
is a categorical variable indicating the class.
Brink {\it et al}. (2012) and Richards {\it et al}. (2012b) consider this
diversity of features in designing automatic classifiers on static data sets. 
They use cross-validation to
find that some features are more useful than others and that the inclusion
of all features degrades the performance of their classifier.  This
determination of optimal inputs is at the heart of machine learning.  Our
intention is to introduce an information-theoretic approach 
based on the output covariance matrix of equation \ref{eq:cov} which will allow
us to perform a similar determination in real time as observations are made,
directing experiments as they are performed.  This sort of information theoretic
approach has already been tried in the static dataset case by Huisje {\it et al}. 
(2011, 2012) who use a the correntropy to determine a light curve's 
true period.  The method should be successful when extended to the case of
multiple observed and latent quantities.

The active learning problem for transient objects contains three
subproblems, active learning, active search, and active feature
acquisition all of which must be made in an online, streaming fashion.  
Rather than
considering an entire pool of test objects, they appear one at a time as
they are detected and the algorithm must decide whether and how to follow
up on each immediately as they are detected.  We first describe algorithms
for each of the three pieces and then how to combine them into a single
algorithm for streaming transient detections.

{\bf Active Learning.} In order to get a ground-truth class for a transient
object, repeated observations are taken to estimate a full light curve.  A
human expert then assigns a class label.  Both the repeated telescope
observations and human time are expensive and thus we want to learn a good
classification model using limited training data.  We propose a
corresponding trace and survey criteria for active learning on transients
as in Section \ref{sec:mlpz}.

{\bf Active Feature Acquisition.}  We propose to extend our recent work on 
using Gaussian Processes to detect damped
lyman-alpha (DLA) systems \cite{Garnett12a}.  In that work GP regression
was used on each observed, noisy spectrum to infer the latent spectrum.
The single independent (input) variable was wavelength.  For transient
objects we will have 5 color magnitudes that are a function of time and are
coupled to each other.  In the DLA work, a different model was learned for
spectra with and without a DLA.  DLAs were classified by recognizing which
model fit best.  In the proposed work, we will learn a different model for
each class of object and will estimate class probabilities using Bayes rule
for combining the prior probability for each class and how well the
respective models fit the observations.  The key advantage of this approach
is that the GPs naturally provide a mean and covariance for future
unobserved colors.  This uncertainty propagates through to class labels and
we can use it to estimate the reduction in class uncertainty that will be
gained by observing a certain color at a certain time.  The observation
yielding the greatest reduction in entropy for the class and the light
curves for this object will be taken.

{\bf Active Search.} Many transient objects will be from common and/or 
well-understood classes that do not have much observational value.  The ultimate
objective in following up detected transients is to maximize the number of
interesting transients classified and characterized while staying within a
budget of follow-up observations.  This is an active search problem. The
problem and the Bayesian optimal algorithm for it are described in our
recent work \cite{Garnett11,Garnett12}.  As in active learning, the
acquisition of class labels is expensive and we need to learn a model to
predict these labels from limited input data.  However, the final
performance objective is not the accuracy of the classifier, but rather the
number of positives (i.e. objects from interesting classes) identified.  We
propose to use the simple myopic algorithm described in that work.  It
computes the probability of each point belonging to the positive class and
chooses the largest.


{\bf A combined streaming method.} The three goals above will be combined
in a staged set of decisions.  When a new transient is detected, the light
curve model will be used to provide estimated light curves for the object.
Those light curves are the input variables for this object in the active
learning and active search algorithms.  In parallel, the active learning
and active search methods will decide whether to follow up on this object.
If either of them selects the object, it is advanced to active feature
acquisition.  There additional observations on the object are selected and
the process for this object repeats.  An object that initially seemed
interesting to one algorithm may cease to be so after additional
observations or may be adopted by the other one.  The process for one
object terminates when neither active learning nor active search remains
interested in it or the object class and light curves are characterized
well enough that no more observations are required.

The active search and active learning algorithms each compute an objective
criterion score for each object.  For photometric redshifts the scores are
used to create a ranked list and observations are scheduled proceeding down
the list.  When a budget is given for follow-ups on each batch of newly
detected transients, going down a ranked list in each batch until that
budget is exhausted is appropriate.

However, when the budget is an aggregate over a longer time period a
streaming decision on how much of the budget to spend on each batch must be
made on that batch in isolation.  This will be done by choosing a score
threshold.  The threshold will be set by evaluating the historical stream
and setting it at a value that would yield a number of follow ups equal to
an available budget for following up.  The threshold will be adjusted
continuously as more observations are taken and the models and scientific
goals change (e.g. the object types designated as ``interesting'' are
changed).

Since all of the algorithm components are based on GP models, it is
possible to merge them together into a single model.  We hypothesize that
additional performance improvements can be made through an integrated model
and decision algorithm.  For example, an object with a modest active
learning score that can be easily characterized with only one more follow
up might be promoted over one with a higher score that can not be easily
characterized even with many more observations.  After we have implemented
the staged method described above, we will investigate and compare an
integrated model.



\subsection{Active learning for calibrating cosmology}
\label{sec:photoz}

The accurate determination of an object's distance or redshift is
central to every test of cosmology that happens outside of a particle
accelerator.  The comparison of redshifts and luminosities of standard
candles enabled the discovery of dark energy and cosmic acceleration
\cite{perlmutter1998}. Redshifts serve as a proxy for radial distance
from Earth to the observed object.  Redshifts thus are necessary for
building three dimensional maps of the distribution of galaxies in the
Universe.  Such maps will help and have helped us to constrain how
galaxies formed over the history of the Universe, and thus can tell us
much about how gravity operates at the largest scales and what the
parameters are that govern the behavior of dark energy and the cosmic
acceleration \cite{muvarpi2,roland,sudeep}.  Accurately determining the redshifts of
distant galaxies is a requirement if we are to answer some of the most
vexing problems in fundamental physics today.

Direct spectroscopic redshift measurements of enough galaxies to
constrain dark energy parameters to the precision required by next
generation experiments would be thousands of times more expensive than
taking the corresponding photometric (or imaging) data.  Large digital
cameras (e.g.\ the 3.2 Gigapixel camera for the LSST) can observe
$\sim 10^6$ sources every 15 seconds (several orders of magnitude more
efficient that spectroscopic observations). Our task, then, is to
construct algorithms whereby we can convert these much cheaper
photometric data into accurate redshifts (i.e. photometric redshifts).

Photometric redshifts are principally determined using
forward-fitting models.  Astronomers assume that they can model the rest frame
spectra of any galaxy.  These spectral models are
redshifted and integrated over the profile of an experiment's
photometric filters until a good fit to the observed photometric data is
found.  The redshift of the galaxy is taken as that which produces the best fit
between template and data.  Many publicly available codes such as 
EAZY \cite{eazy} implement this method.  
While it is straightforward in principle, it requires
accurate foreknowledge to select the appropriate basis spectra.  If the chosen
spectra are not representative of the population of observed
galaxies, the algorithm will fail to give accurate redshifts and cosmological
inferences will be inaccurate \cite{budavari2008}.  The effects of this
shortcoming can be seen in Figure \ref{subfig:eazy}, which plots the results
of running the publically available EAZY algorithm \cite{eazy} on a set of
simulated galaxy observations designed to represent results expected from the
Large Synoptic Survey Telescope (LSST\footnote{http://www.lsst.org}).
While many of the galaxies fall near the
$z_\text{photometric}=z_\text{spectroscopic}$ line, there is significant scatter
in the results.  We propose to overcome this difficulty with an exclusively
data-driven algorithm based on Gaussian Processes.

\begin{figure}
\subfigure[]{
\includegraphics[scale=0.3]{eazy_scatter_plot.png}
\label{subfig:eazy}
}
\subfigure[]{
\includegraphics[scale=0.3]{gp_scatter_plot.png}
\label{subfig:gp}
}
\caption{
Photometric redshift plotted
against true spectroscopic redshift for 48,000 simulated LSST galaxy
observations.  Photometric redshifts are derived using the
EAZY template-fitting
algorithm in Figure \ref{subfig:eazy} and
our Gaussian Process based algorithm in Figure \ref{subfig:gp}.
}
\label{fig:scatter}
\end{figure}

Other works have already attempted to apply Gaussian Processes to the problem of
photometric redshfifts \cite{kaufman,bonfield}, however, they have treated the
problem as one of learning the form of a one-to-one scalar function.  We propose
to use the probabilistic nature of Gaussian Processes to learn the full
probability distribution that a given galaxy is at a given redshift.  
This will produce an algorithm that is simultaneously more robust against
sparse, noisy or degenerate training data and more amenable to improvement 
by the introduction of
active learning.  We describe our method below.

We will treat the photometric redshift problem probabilistically. 
We will use Gaussian Processes to model a probability density
function in $P(z_\text{photometric})$ which can characterize the probability that a
given galaxy is at a given redshift.  This method will allow us
to quantify the information content of our trainind data in such a way as to
optimize the gathering of additional spectroscopic data to further improve on
our results.  
Figure \ref{subfig:gp} shows preliminary results from our
algorithm when trained on spectroscopic data from 50,000 galaxies and tested
on the same 48,000 galaxies as Figure \ref{subfig:eazy}.
%We discuss specifically how we model $P(z_\text{photometric})$ below.

%Galaxy observations are represented as vectors $\{\vec{\theta}_q\}$ of
%flux information.  For each unknown
%test galaxy $\{\vec{\theta}_q\}$, we find the $k$ nearest neighbor training
%galaxies in flux space ($k$ is treated as a parameter to be optimized by our
%algorithm).  We divide these neighbor galaxies into two sub-populations based on
%their spectroscopic redshifts and fit a Gaussian Process to each sub-population.
%This gives us a bi-modal probability distribution for $P(z_\text{photometric})$. 
%We compare the likelihood of this hypothesis to a single-mode
%$P(z_\text{photometric})$ in which all of the neighbor galaxies are fit together
%in one Gaussian Process.  The final $P(z_\text{photometric})$ is a linear
%combination of these two hypotheses, weighted according to their respective
%model likelihoods.

\begin{figure}
\subfigure[]{
\includegraphics[scale=0.3]{lnsum_lsst.png}
\label{subfig:lnsumlsst}
}
\subfigure[]{
\includegraphics[scale=0.3]{sdss_lnsum.png}
\label{subfig:lnsumsdss}
}
\caption{
Figure \ref{subfig:lnsumlsst} shows the mean value of
$\ln[P(\text{truth})]$ as a function of photometric redshift (the vertical
axes in Figures \ref{fig:scatter}) for all
three algorithms under consideration.  
Figure \ref{subfig:lnsumsdss} compares our Gaussian Process method
to the artificial neural network code ANNz \cite{annz} on real
data taken from the Sloan Digital Sky Survey \cite{Abazagian:2008wr}.  
In this latter case, the
algorithms are trained on 70,000 galaxies and tested on 715,000 galaxies.
}
\label{fig:gp}
\end{figure}

Other data-driven algorithms for photometric redshift determination do exist. 
An example of these is the publically-available code ANNz \cite{annz},
which is based on an artificial neural network scheme.
In this case, the principal
shortcoming the method is that the artificial neural network
is designed only to return only a
photometric redshift value and an uncertainty.  This leaves its results
sensitive to degeneracies in the photometric data whereby low redshift galaxies
look similar to high redshift galaxies, confusing the algorithm. 
Figure \ref{fig:gp} plots the mean value of $\ln[P(\text{truth})]$,
i.e. the value of the logarithm $P(z_\text{photometric})$ at the point
$z_\text{photometric}=z_\text{spectroscopic}$ 
as a function
of photometric redshift for EAZY, ANNz, and our Gaussian Process algorithm.
We see that both EAZY and ANNz consistenly assign lower probabilities to
$z_\text{photometric}=z_\text{spectroscopic}$ than does our Gaussian Process algorithm.

The demands of next generation cosmological experiments will require
that our photometric redshift determinations be accurate to within
$\le 2\times 10^{-3}(1+z)$ \cite{desc}.  This is a hard limit, as a
bias in redshift determination of just 0.01 can degrade dark energy
constraints by as much as 50\%
\cite{kitching,huterer2006,nakajima2011}.  Testing present template
and empirical methods on a sample of 5,482 galaxies from the 2df-SDSS
LRG and Quasar survey, Abdalla {\it et al}. (2011) find biases of
order 0.05 (see their Figure 4).  This level of bias can degrade dark
energy constraints by as much as a factor of 3 \cite{Ma2006}.
Considering 3,000 galaxies from the DEEP2 EGS and zCOSMOS surveys and
using Bayesian methods, Mandelbaum {\it et al}.  (2008) find a bias in
redshift determination of order 0.01 (see their Table 2).  While this
is an improvement, it still an order of magnitude larger than what is
required.


How can we resolve these problems?  In both the empirical and template
photometric redshift codes, biases arise from the fact that the
training samples (templates) do not occupy the same color and redshift
space as the data.  Improving on this through targeted observations
is, however, expensive.  Simple sampling strategies (e.g. random or
stratefied) are not efficient.  We need a technique to identify the
next best observation to take to best reduce the redshift estimation
bias of our algorithm.  Active learning will provide this technique.

Focus on these as our primary test to develop these algorithms


%The computational and statistical challenges for searching for optimal
%classifications or sets of experiments are
%exacerbated by the usual desire to construct a model over the {\em entire}
%state space of a system.  For example, when learning classifiers, we
%typically score the performance based on prediction accuracy over an entire
%test set.  It is less common to offer the system a chance to browse the
%test set and report whatever interesting observations it can make about it.
%The problem becomes worse when we attempt to follow model learning with
%discovery.  Optimizing over learned models is always risky since the
%optimizer is likely to find the learner's mistakes.  


%Small mistakes in the learning process can lead to big mistakes
%in identifying anomalies.  For example, suppose a structure learning
%algorithm fails to include a link between two variables that are
%highly related at least for a subset of their possible values.  All
%records that are anomalies based on an unlikely combination of those
%attributes will be missed.  Similarly, a spurious additional link with
%poorly learned parameters will create anomalies where there are none.
%Mistakes like these are inevitable in large models.


Information gain and astronomy (Jake's stuff)

\section{Conclusion}

We probably need one of these....

\begin{thebibliography}{99}

\bibitem[Abazajian {\it et al}. 2009]{Abazajian:2008wr}
  Abazajian, K.~N.{\it et al.}  [SDSS Collaboration]~2009,
  %``The Seventh Data Release of the Sloan Digital Sky Survey,''
  The Astrophysical Journal Supplement Series  {\bf 182}, 543
  [arXiv:0812.0649 [astro-ph]].
  %%CITATION = APJSA,182,543;%%

\bibitem[Abdalla {\it et al}. 2011]{abdalla}
Abdalla,~F.~B., Banerji,~M., Lahav,~O., and Rashkov,~V. 2011,
Monthly Notices of the Royal Astronomical Society {\bf 417}, 1891

\bibitem[Abramo {\it et al}. 2012]{narrow}
Abramo,~L.~R., Strauss,~M.~A., Lima,~M., Hern\'andez-Monteagudo,~C., Lazkoz,~R.,
Moles,~M., de Oliveira,~C.~M., Sendra,~I., Sodr\'e Jr.,~L., and
Storchi-Bergmann,~T. 2012, Monthly Notices of the Royal Astronomical Society
{\bf 423}, 3251

\bibitem[Albrecht {\it et al}. 2006]{detf}
Albrecht,~A., Bernstein,~B., Cahn,~R., Freedman,~W.~L., Hewitt,~J.,
Hu,~W., Huth,~J., Kamionkowski,~M., Kolb,~E., Knox,~L., Mather,~J.~C.,
Staggs,~S., Suntzeff,~N.~B. (Dark Energy Task Force) 2006,
``Report of the Dark Energy Task Force,''
\verb|http://jdem.gsfc.nasa.gov/science/DETF_Report.pdf|


\bibitem[Berg\'e {\it et al}. 2012]{psf}
Berg\'e,~J., Price,~S., Amara,~A., and Rhodes,~J. 2012,
Monthly Notices of the Royal Astronomical Society {\bf 419}, 2356

\bibitem[Blocker and Protopapas 2013]{blocker2013}
Blocker, Alexander W. and Protopapas, Pavlos 2013, arXiv:1301.3027

\bibitem[Bloom {\it et al}. 2011]{bloom2011}
Bloom,~J.~S., Richards,~J.~W., Nugent,~P.~E., Quimby,~R.~M., Kasliwal,~M.~M.,
Starr,~D.~L., Posnanski,~D., Ofek,~E.~O., Cenko,~S.~B., Butler,~N.~R.,
Kulkarni,~S.~R., Gal-Yam,~A., and Law,~N. 2011 [arXiv:1106.5491]

\bibitem[Bonfield {\it et al}. 2010]{bonfield}
Bonfield,~D.~G., Sun,~Y., Davey,~N., Jarvis,~M.~J., Abdalla,~F.~B.,
Banerji,~M., Adams,~R.~G. 2010, Monthly Notices of the Royal Astronomical Society 
{\bf 405} 987

\bibitem[Brammer {\it et al}. 2008]{eazy}
Brammer,~G.~B., van Dokkum,~P.~G., and Coppi,~P. 2008,
The Astrophysical Journal {\bf 686}, 1503

\bibitem[Brink {\it et al}. 2012]{brink2012}
Brink, Henrik, Richards, Joseph W., Poznanski, Dovi, Bloom, Joshua S., Rice,
John, Negahban, Sahand, and Wainwright, Martin 2012, arXiv:1209.3775

\bibitem[Bryan 2007]{brentsthesis}
Bryan, B., 2007, Ph.D. thesis
\verb|http://reports-archive.adm.cs.cmu.edu/anon/|
\verb|ml2007/abstracts/07-122.html|

\bibitem[Bryan {\it et al}. 2007]{bryan}
Bryan, B., Schneider, J., Miller, C.~J., Nichol, R.~C., Genovese, C., and
Wasserman, L., 2007,
The Astrophysical Journal {\bf 665}, 25

\bibitem[Budav\'ari 2008]{budavari2008}
Budav\'ari,~T. 2008 The Astrophysical Journal {\bf 695}, 747

\bibitem[Collister and Lahav 2004]{annz}
Collister,~A.~A. and Lahav,~O. 2004,
Publications of the Astronomical Society of the Pacific {\bf 116}, 345

\bibitem[Connolly {\it et al}. 2005]{imsim}
Connolly,~A.~J., Peterson,~J., Jernigan,~J.~G., Abel,~R., Bankert,~J.,
Chang,~C., Claver,~C.~F., Gibson,~R., Gilmore,~D.~K., Grace,~E., Jones,~R.~L.,
Ivezic,~Z., Jee,~J., Juric,~M., Kahn,~S.~M., Krabbendam,~V.~L., Krughoff,~S.,
Lorenz,~S., Pizagno,~J., Rasmussen,~A., Todd,~N. Tyson,~J.~A., and Young,~M.
2005, Society of the Photo-Optical Instrumentation Engineers (SPIE) Converence
Series {\bf 7738}, 53

\bibitem[Cunha {\it et al}. 2012]{cunha2012}
Cunha,~C.~E., Huterer,~D., Lin,~H., Busha,~M.~T., and Wechsler,~R.~H. 2012,
[arXiv:1207.3347]

\bibitem[Daniel and Linder 2010]{muvarpi2}
Daniel,~S.~F. and Linder,~E.~V. 2010, Physical Review D {\bf 82}, 103523

%\cite{Daniel:2011rr}
\bibitem[Daniel {\it et al}. 2011]{daniel2011} 
  Daniel,~S.~F., Connolly,~A.~J., Schneider,~J., Vanderplas,~J. and Xiong,~L. 
  %``Classification of Stellar Spectra with LLE,''
  The Astronomical Journal  {\bf 142}, 203 (2011)
  [arXiv:1110.4646 [astro-ph.SR]].
  %%CITATION = ARXIV:1110.4646;%%

\bibitem[Daniel {\it et al}. 2012]{daniel2012}
Daniel,~S.~F., Connolly,~A.~J., and Schneider,~J. 2012
[arXiv:1205.2708]

\bibitem[Das {\it et al}. 2011]{sudeep}
Das,~S., de Putter,~R., Linder,~E.~V., and Nakajima,~R. 2011,
[arXiv:1102.5090]

\bibitem[Davis {\it et al}. 2007]{essence}
Davis,~T.~M., M\"ortsell,~E., Sollerman,~J., Becker,~A.~C., Blondin,~S.,
Challis,~P., Clocchiatti,~A., Filippenko,~A.~V., Foley,~R.~J., Garnavich,~P.~M.,
Jha,~S., Krisciunas,~K., Kirshner,~R.~P., Leibundgut,~B., Li,~W., Matheson,~T.,
Miknaitis,~G., Pignata,~G., Rest,~A., Riess,~A.~G., Schmidt,~B.~P.,
Smith,~R.~C., Spyromilio,~J., Stubbs,~C.~W., Suntzeff,~N.~B., Tonry,~J.~L.,
Wood-Vasey,~W.~M., and Zenteno,~A. 2007, The Astrophysical Journal, {\bf 666},
716

\bibitem[de Putter {\it et al}. 2010]{roland}
de Putter,~R., Huterer,~D. and Linder,~E.~V. 2010, Physical Review D {\bf 81},
103513


\bibitem[Djorgovski {\it et al}. 2011]{djorgovski2011}
Djorgovski,~S.~J., Donalek,~C., Mahabal,~A.~A., Moghaddam,~B., Turmon,~M.,
Graham,~M.~J., Drake,~A.~J., Sharma,~N. and Chen,~Y. 2011
[arXiv:1110.4655] to appear in Statistical Analysis and Data Mining, ref. proc.
CIDU 2011 conf., eds. A. Srivastava and N. Chawla

\bibitem[Garnett {\it et al}. 2011]{Garnett11}
Garnett,~R., Krishnamurhty,~Y., Wang,~D., Schneider,~J., and Mann,~R. 2011,
``Bayesian Optimal Active Search on Graphs,'' KDD Workshop on Mining and
Learning with Graphs

\bibitem[Garnett {\it et al}. 2012a]{Garnett12}
Garnett,~R., Krishnamurthy,~Y., Xiong,~X., Schneider,~J., and Mann,~R. 2012a,
``Bayesian Optimal Active Search and Surveying,'' International Conference on
Machine Learning

\bibitem[Garnett {\it et al}. 2012b]{Garnett12a}
Garnett,~R., Ho,~S., and Schneider,~J. 2012b,
``Gaussian Processes for Identifying Damped Lyman-alpha Systems in Spectroscopic
Surveys,'' Neural Information Processing Systems 
workshop on Modern Nonparametric Methods in Machine Learning

\bibitem[Graham {\it et al}. 2012]{graham2012}
Graham,~M.~J., Djorgovski,~S.~G., Mahabal,~A., Donalek,~C., Drake,~A.,
Longo,~G. 2012 [arXiv:1208.2480] to appear in special issue of Distributed and
Parallel Databases on Data Intensive eScience

\bibitem[Huijse {\it et al}. 2011]{huijse2011}
Huijse, Pablo, Est\'evez, Pablo A., Zegers, Pablo, Pr\'incipe, Jose C.,
and Protopapas, Pavlos 2011, IEEE Signal Processing Letters {\bf 18}, 371

\bibitem[Huijse {\it et al}. 2012]{huijse2012}
Huijse, Pablo, Est\'evez, Pablo A., Protopapas, Pavlos, Zebers, Pablo,
and Pr\'incipe, Jos\'e C. 2012, arXiv:1212.2398

\bibitem[Huterer {\it et al}. 2006]{huterer2006}
Huterer,~D., Takada,~M., Bernstein,~G., and Jain,~B. 2006,
Monthly Notices of the Royal Astronomical Society {\bf 366}, 101

\bibitem[Kaufman {\it et al}. 2011]{kaufman}
Kaufman, Cari G., Bingham, Derek, Habib, Salman, Heitmann, Katrin, and Frieman,
Joshua A. 2011, The Annals of Applied Statistitcs {\bf 5} 2470

\bibitem[Kim {\it et al}. 2011]{kim2011}
Kim, Dae-Won, Protopapas, Pavlos, Byun, Young-Ik, Alcock, Charles, Khardon,
Roni, and Trichas, Markos 2011, arXiv:1101.3316

\bibitem[Kitching {\it et al}. 2008]{kitching}
Kitching,~T.~D., Taylor,~A.~N., and Heavens,~A.~F. 2008,
Monthly Notices of the Royal Astronomical Society {\bf 389} 173

\bibitem[Long {\it et al}. 2012]{long2012}
Long,~J.~P., El Karoui,~N., Rice,~J.~A., Richards,~J.~W., and Bloom,~J.~S. 2012,
Publications of the Astronomical Society of the Pacific {\bf 124} 280

\bibitem[LSST Collaboration 2011]{lsstoverview}
LSST Collaboration 2011, [arXiv:0805.2366]
\verb|http://www.lsst.org/lsst/overview/|

\bibitem[LSST Dark Energy Science Collaboration 2012]{desc}
LSST Dark Energy Science Collaboration 2012, [arXiv:1211.0310]

\bibitem[LSST Science Collaborations 2009]{sciencebook}
LSST Science Collaborations 2009, ``LSST Science Book'',
\verb|http://www.lsst.org/lsst/science/scibook|

\bibitem[Ma {\it et al}. 2006]{Ma2006}
Ma,~Z., Hu,~H., and Huterer,~D. 2006, The Astrophysical Journal {\bf 636}, 21

\bibitem[Ma {\it et al}. 2012]{YifeiMa12}
Ma,~Y., Garnett,~R., and Schneider,~J. 2012,
``Submodularity in Batch Active Learning and Survey Problems
on Gaussian Random Fields,''
Neural Information Processing Systems 
workshop on Discrete Optimization in Machine Learning

\bibitem[Mahabal {\it et al}. 2008a]{mahabal2008a}
Mahabal,~A., Djorgovski,~S.~G., Turmon,~M., Jewell,~J., Williams,~R.~R.,
Drake,~A.~J., Graham,~M.~G., Donalek,~C., Glikman,~E., and the Palomar-QUEST Team
2008a, Astronomische Nachrichten {\bf 329}, 288

\bibitem[Mahabal {\it et al}. 2008b]{mahabal2008b}
Mahabal,~A., Djorgovski,~S.~G., Williams,~R., Drake,~A., Donalek,~C.,
Graham,~M., Moghaddam,~B., Turmon,~M., Jewell,~J., Khosla,~A., and
Hensley,~B. 2008b [arXiv:0810.4527] to appear in proceedings fo the Class2008
conference (Classification and Discovery in Large Astronomical Surveys, Ringberg
Castle, 14-17 October 2008)

\bibitem[Mahabal {\it et al}. 2011a]{mahabal2011a}
Mahabal,~A.~A., Donalek,~C., Djorgovski,~S.~J., Drake,~A.~J.,
Graham,~M.~J., Williams,~R., Chen,~Y., Moghaddam,~B., and Turmon,~M.
2011a, [arxiv:1111.3699] to appear in Proc. IAU 285, ``New Horizons in Transient
Astronomy,'' Oxford, September 2011

\bibitem[Mahabal {\it et al}. 2011b]{mahabal2011b}
Mahabal,~A.~A., Djorgovski,~S.~G., Drake,~A.~J., Donalek~C., Graham,~M~J.,
Williams,~R.~D., Chen,~Y., Moghaddam,~B., Turmon,~M., Beshore,~E., and Larson,~S.
2011b, Bulletin of the Astronomical Society of India {\bf 39}, 387

\bibitem[Mandelbaum {\it et al}. 2008]{mandelbaum2008}
Mandelbaum,~R., Seljak,~U., Hirata,~C.~M., Bardelli,~S., Bolzonella,!M.,
Bongiorno,~A., Carollo,~M., Contini,~T., Cunha,~C.~E., Garilli,~B.,
Iovino,~A., Kambczyk,~P, Kneib,~J.-P., Knobel,~C., Koo,~D.~C., Lamareille,~F.,
Le F\`evre,~O., Leborgne,~J.-F., Lilly,~S.~J., Maier,~C., Mainieri,~V.,
Mignoli,~M., Newman,~J.~A., Oesch,~P.~A., Perez-Montero,~E., Ricciardelli,~E.,
Scodeggio,~M., Silverman,~J., and Tasca,~L. 2008, Monthly Notices of the Royal
Astronomical Society {\bf 386}, 781

\bibitem[McBride {\it et al}. 2011a]{mcbride2011a}
McBride,~C.~K., Connolly,~A.~J., Gardner,~J.~P., Scranton,~R., Newman,~J.~A.,
Scoccimarro,~R., Zehavi,~I., and Schneider,~D.~P. 2011a, The Astrophysical
Journal, {\bf 726}, 13

\bibitem[McBride {\it et al}. 2011b]{mcbride2011b}
McBride,~C.~K., Connolly,~A.~J., Gardner,~J.~P., Scranton,~R., Scoccimarro,~R.,
Berlind,~A.~A., Mar\'in,~F., and Schneider,~D.~P. 2011b, The Astrophysical
Journal {\bf 739}, 85

\bibitem[Moore {\it et al}. 2000]{Moore00}
Moore,~A., Connolly,~A., Genovese,~C., Grone,~L., Kanidoris,~N., Nichol,~R.,
Schneider,~J., Szalay,~A., Szapudi,~I., and Wasserman,~L. 2000,
``Fast Algorithms and Efficient Statistics: N-point Correlation Functions,'' in
MPA/MPE/ESO Conference on Mining the Sky [arXiv:astro-ph/0012333]

\bibitem[Morgan {\it et al}. 2011]{morgan2011}
Morgan, A.N., Long, James, Richards, Joseph W., Broderick, Tamara, Butler,
Nathaniel R., and Bloom, Joshua S. 2011, arXiv:1112.3654

\bibitem[Nakajima {\it et al}. 2012]{nakajima2011}
Nakajima,~R., Mandelbaum,~R., Seljak,~U., Cohn,~J.~D., Reyes,~R., and
Cool,~R. 2012, Monthly Notices of the Royal Astronomical Society {\bf 420}, 3240
[arXiv:1107.1395]

\bibitem[Nichol {\it et al}. 2006]{Nichol2006}
Nichol,~R.~C., Sheth,~R.~K., Suto,~Y., Gray,~A.~J., Kayo,~I., Wechsler,~R.~H.,
Marin,~F., Kulkarni,~G., Blanton,~M., Connolly,~A.~J., Gardner,~J.~P., Jain,~B.,
Miller,~C.~J., Moore,~A.~W., Pope,~A., Pun,~J., Schneider,~D., Schneider,~J.,
Szalay,~A., Szapudi,~I., Zehavi,~I., Bahcall,~N.~A., Csabai,~I., Brinkmann,~J.
2006, Monthly Notices of the Royal Astronomical Society {\bf 368}, 1507

\bibitem[Oluseyi {\it et al.} 2012]{rrlyrae}
Oluseyi, Hakeem M., Becker, Andrew C., Culliton, Christopher, Furqan, Muhammad,
Hoadley, Keri L., Regencia, Paul, Wells, Akeem J., Ivez\`ic, Zeljko, Jones, R.
Lynne, Krughoff, K. Simon, and Sesar, Branimir (2012), The Astronomical Joural
{\bf 144} 9

\bibitem[Pichara {\it et al}. 2012]{pichara2012}
Pichara, K., Protopapas, P., Kim, D.-W., Marquette, J.-B., and Tisserand, P.
2012, Monthly Notices of the Royal Astronomical Society, {\bf 427} 1284

\bibitem[Poczos and Schneider 2011]{poczos11alphadiv}
Poczos,~B. and Schneider,~J. 2011, ``On the Estimation of alpha-Divergences,''
Artificial Intelligence and Statistics (AISTATS)

\bibitem[Poczos {\it et al}. 2011]{Poczos2011UAI}
Poczos,~B., Xiong,~L., and Schneider,~J. 2011, ``Nonparametric Divergence Estimation with
Applications to Machine Learning on Distributions,''  Uncertainty in Artificial
Intelligence

\bibitem[Poczos {\it et al}. 2012]{poczos12CVPR}
Poczos,~B., Xiong,~L., Sutherland,~D., and Schneider,~J. 2012,
``Nonparametric Kernel Estimators for Image Classification,''
IEEE Conference on Computer Vision and Pattern Recognition

\bibitem[Preston {\it et al}. 2009]{preston2009}
Preston, Dan, Protopapas, Pavlos, and Brodely, Carla 2009, arXiv:0901.3329

\bibitem[Rasmussen and Williams 2006]{gp}
Rasmussen, C.~E. and Williams, C.~K.~I., 2006, ``Gaussian
Processes for Machine Learning''
\verb|http://www.GaussianProcess.org/gpml/|

\bibitem[Richards {\it et al}. 2004]{qso}
Richards,~G.~T., Nichols,~R.~C., Gray,~A.~G., Brunner,~R.~J., Lupton,~R.~H.,
Vanden Berk,~D.~E., Chong,~S.~S., Weinstein,~M.~A., Schneider,~D.~P.,
Anderson,~S.~F., Munn,~J.~A., Harris,~H.~C., Strauss,~M.~A., Fan,~X.,
Gunn,~J.~E., Ivezi\'c,~Z., York,~D.~G., Brinkmann,~J., and Moore,~A.~W. 2004,
The Astrophysical Journal Supplement Series, {\bf 155}, 257

\bibitem[Richards {\it et al}. 2011]{richards2011}
Richards,~J.~W., Starr,~D.~L., Butler,~N.~R., Bloom,~J.~S., Brewer,~J.~M.,
Crellin-Quick,~A., Higgins,~J., Kennedy,~R., and Rischard,~M. 2011,
The Astrophysical Journal {\bf 733}, 10

\bibitem[Richards {\it et al}. 2012a]{richards2012}
Richards,~J.~W., Starr,~D.~L., Brink,~H., Miller,~A.~A., Bloom,~J.~S.,
Butler,~N.~R., James,~J.~B., Long,~J.~P., and Rice,~J. 2012a
The Astrophysical Journal {\bf 744}, 192

\bibitem[Richards {\it et al}. 2012b]{richards2012b}
Richards, Joseph W., Starr, Dan L., Miller, Adam A., Bloom, Joshua S.,
Butler, Nathaniel R., Brink, Henrik, and Crellin-Quick, Arien 2012b,
The Astrophysical Journal Supplement Series {\bf 203} 32

\bibitem[Rosenfield {\it et al}. 2011]{rosenfield2011}
Rosenfield,~P., Connolly,~A., Fay,~J., Sayres,~C., and Tofflemire,~B. 2011,
Astronomical Society of the Pacific Conference Series {\bf 443}, 109

\bibitem[Scranton {\it et al}. 2002]{Scranton2002}
Scranton,~R., Johnston,~D., Dodelson,~S., Frieman,~J.~A., Connolly,~A.,
Eisenstein,~D.~J., Gunn,~J.~E., Hui,~L., Jain,~B., Kent,~S., Loveday,~J.,
Narayanan,~V., Nichol,~R.~C., O'Connell,~L., Soccimarro,~R., Sheth,~R.~K.,
Stebbins,~A., Strauss,~M.~A., Szalay,~A.~S., Sapudi,~I., Tegmark,~M.,
Vogeley,~M., Zehavi,~I., Annis,~J., Bahcall,~N.~A., Brinkman,~J., Csabai,~I.,
Hindsley,~R., Ivezic,~Z., Kim,~R.~S.~J., Knapp,~G.~R., Lamb,~D.~Q., Lee,~B.~C.,
Lupton,~R.~H., McKay,~T., Munn,~J., Peoples,~J., Pier,~J., Richards,~G.~T.,
Rockosi,~C., Schlegel,~D., Schneider,~D.~P., Stoughton,~C., Tucker,~D.~L.,
Yanny,~B., York,~D.~G. 2002, The Astrophysical Journal {\bf 579}, 48

\bibitem[Sesar {\it et al}. 2011]{linear}
Sesar,~B., Stuart,~J.~S., Ivezi\'c,~\u Z., Morgan,~D.~P., Becker,~A.~C., and
Wo\'zniak,~P. 2011, The Astronomical Journal {\bf 142}, 190

\bibitem[Settles 2009]{activelearning}
Settles,~B. 2009, ``Active Learning Literature Survey,'' Computer Sciences Technical
Report 1648, University of Wisconsin-Madison,
\verb|http://pages.cs.wisc.edu/~bsettles/active-learning/|


\bibitem[Shafieloo {\it et al}. 2012]{ericgp}
Shafieloo,~A., Kim,~A.~G., and Linder,~E.~V. 2012,
Physical Review D {\bf 85}, 123530 [arXiv:1204.2272]


\bibitem[Skibba {\it et al}. 2006]{Skibba2006}
Skibba,~R., Sheth,~R.~K., Connolly,~A.~J., and Scranton,~R. 2006,
Monthly Notices of the Royal Astronomical Society, {\bf 369}, 68

\bibitem[Straf 2003]{straf03}
Straf,~M.~L. 2003, Journal of the American Statistical Association {\bf 98}, 1

\bibitem[Szapudi {\it et al}. 2002]{Szapudi2002}
Szapud,~I., Frieman,~J.~A., Scoccimarro,~R., Szalay,~A.~S., Connolly,~A.~J.,
Dodelson,~S., Eisenstein,~D.~J., Gunn,~J.~E., Johnston,~D., Kent,~S.,
Loveday,~J., Meiksin,~A., Nichol,~R.~C., Scranton,~R., Stebbins,~A.,
Vogeley,~M.~S., Annis,~J., Bahcall,~N.~A., Brinkman,~J., Csabai,~I., Doi,~M.,
Fukigita,~M., Ivezi\'c,~\u Z., Kim,~R.~S.~J., Knapp,~G.~R., Lamb,~D.~Q.,
Lee,~B.~C., Lupton,~R.~H., McKay,~T.~A., Munn,~J., Peoples,~J., Pier,~J.,
Rockosi,~C., Schlegel,~D., Stoughtfon,~C., Tucker,~D.~L., Yanny,~B., York,~D.~G.
2002, The Astrophysical Journal {\bf 570}, 75

\bibitem[Vanderplas and Connolly 2009]{vdp2009}
Vanderplas,~J. and Connolly,~A.~J. 2009,
Astronomical Journal {\bf 138}, 1365

\bibitem[Wang {\it et al}. 2011]{wang2011}
Wang, Yuyang, Khardon, Roni, and Protopapas, Pavlos 2011
arXiv:1111.1315

\bibitem[Wang {\it et al}. 2012]{wang2012}
Wang, Yuyang, Khardon, Roni, and Protopapas, Pavlos 2012
arXiv:1203.0970

\bibitem[Wiley {\it et al}. 2011]{wiley2011}
Wiley,~K, Connolly,~A.~J., Gardner,~J., Krughoff,~S., Balazinska,~M., Howe,~B.,
Kwon,~Y., and Bu, ~Y. 2011, Publication of the Astronomical Society of the
Pacific {\bf 123}, 366

\bibitem[Xiong {\it et al}. 2011a]{Xiong2011gad}
Xiong,~L., Poczos,~B., Schneider,~J., Connolly,~A., Vanderplas,~J. 2011a,
``Hierarchical Probabilistic Models for Group Anomaly Detection,''
Artificial Intelligence and Statistics (AISTATS)

\bibitem[Xiong {\it et al}. 2011b]{xiong2011fgm}
Xiong,~L., Poczos,~B., and Schneider,~J. 2011, ``Group Anomaly Detection using Flexible
Genre Models,'' Neural Information Processing Systems

\bibitem[Yip {\it et al}. 2004]{yip2004a}
Yip,~C.~W., Connolly,~A.~J., Szalay,~A.~S., Budav\'ari,~T., SubbaRao,~M.,
Frieman,~J.~A., Nichol,~R.~C., Hopkins,~A.~M., York,~D.~G., Okamura,~S.,
Brinkmann,~J., Csabai,~I., Thakar,~A.~R., Fukugita,~M., 
and Ivezi\'c,~\u Z. 2004, The Astronomical Journal {\bf 128}, 585

\bibitem[Zhang and Schneider 2010a]{YiZhangICML2010}
Zhang,~Y. and Schneider,~J. 2010a, ``Projection Penalties: Dimension Reduction without
Loss,'' International Conference on Machine Learning

\bibitem[Zhang and Schneider 2010b]{YiZhangMultitask2010}
Zhang,~Y. and Schneider,~J. 2010b,
``Learning Multiple Tasks with a Sparse Matrix-Normal Penalty,''
Neural Information Processing Systems

\bibitem[Zhang {\it et al}. 2010]{YiZhangSDM2010}
Zhang,~Y., Schneider,~J., and Dubrawski,~A. 2010,
``Learning Compressible Models,'' Proceedings of SIAM Data Mining Conference

\bibitem[Zhang and Schneider 2011]{YiZhang2011multiECOC}
Zhang,~Y. and Schneider,~J 2011, ``Multi-label Output Codes using Canonical Correlation
Analysis,'' Artificial Intelligence and Statistics

\bibitem[Zhang and Schneider 2012]{YiZhang2012}
Zhang,~Y. and Schneider,~J. 2012, ``Maximum Margin Output Coding,''
International Conference on Machine Learning

\end{thebibliography} 
\label{lastpage}





\end{document}

\subsection{Active Learning and Photometric Redshifts}
\label{sec:mlpz}

The discussion above illustrates how the appropriate choice of machine
learning algorithm can affect the fidelity of one's photometric redshift
determinations.  Further gains can be made if one is similarly judicious in
choosing a training set.  While next generation surveys like the LSST will
be exclusively photometric, they will present us with large numbers of
galaxies which we will have the option to follow-up with off-site
spectroscopy.  It behooves us, therefore, to develop a quantitative way of
determining which galaxies represent the most effective use of these
limited follow-up resources.  Such methods fall in the field of active
learning and we will develop novel algorithms for this purpose.

Active learning algorithms iteratively decide which data points they will
collect outputs on and add to the training set.  The goal is to choose the
points that will most improve the model being learned.  At each step, they
consider the current training data, the potential training data that could
be collected, and the current learned model, and evalute each potential new
point according to some objective criterion.  After selecting one or more
new points, the outputs for those points are collected and added to the
training set and the process repeats.  The key to the active learning
algorithm is the specification of the objective criterion used for
selection.  Settles (2009) presents a survey of such criteria.



Real-time automatic classification of objects is already widely acknowledged as a
necessary support technology for the forthcoming age of survey astronomy
\cite{djorgovski2011,richards2011,richards2012,graham2012,mahabal2008a,mahabal2011a}.
Objects will need to be categorized into known science classes so that novel
or rare objects can be flagged for detailed follow-up observations.
For transient events, algorithms must be able to make rapid
decisions so that sources can be targeted for follow-up 
and classifications learned before
objects return to their quiescent phases.  A great deal of work has
already been done developing algorithms that can learn the classification of an
object given a fixed set of observations and training data.  
Mahabal {\it et al}. (2008a,2011a,2011b) propose to break down the
observations of a given object into $\{\Delta m,\Delta t\}$ pairs (where $m$
is magnitude and $t$ is time) and use the density of observations in this 
two-dimensional space as the basis for a Bayesian classification algorithm.
Mahabal {\it et al}. (2008b) alternatively propose to use those same 
$\{\Delta m,\Delta t\}$ pairs as the input to a Gaussian process regression
by which they will reconstruct the object's entire light curve as a function
of time, and then classify the object based on that reconstruction.
Richards {\it et al}. (2011) use observations of transient objects to extract
periodic (e.g. the amplitude and frequency of the first two Fourier modes of
the object's light curve) and non-periodic (e.g. the variance and skewness of
all of the magnitude observations taken, regardless of their separation in
time) and feed those features into several tree-based classifiers.
They find misclassification rates lower than 30\% with their best method yielding a
misclassification rate of 22.8\%.  Using only non-periodic features, which will be
especially easy for survey telescopes to gather, rather than
full light curves, they 
find a misclassification rate of
between 26\% and 28\%.
Bloom {\it et al}. (2011) also use a tree-based automatic classifier on
Palomar Transient Factory data and find a 3.8\% error rate when
discriminating between four major classifications.  Richards {\it et al}.
consider a more complete set of 25 possible classifications.
Clearly, many possibile approaches are available for the automated
classification of transient objects, and not all of them rely upon highly
detailed observations to function.

While significant attention has been paid to the problem of classifying an
object once training sets and observations have been assembled, relatively
little has been paid to optimizing the training set or observations.
Though the budgeted observing programs of survey telescopes leave little
time to follow-up serendipitous discoveries, other telescopes will be much
freer to fill in the gaps in survey-derived knowledge.  The question is
how best to do so.  The problem of sub-optimal training data is already being
faced by the community.  Figure 6 of Richards {\it et al}. (2011)
demonstrates that automated classifiers are much better at identifying
objects of common types, for which training data is abundant, than they are
at identifying objects of rare types, for which training data is sparse.
The authors address this problem in Richards {\it et al.} (2012a) by designing
an active learning algorithm which selectively adds to the training set,
choosing objects of uncertain classification and asking human classifiers to
provide detailed information (via either their expertise alone or with
follow-up observations) about the queried object.  They find that this scheme
reduces the misclassification rate 
of their classifier by approximately five percentage
points over a scheme which randomly constructs its training data set (see
their Figure 6).

This work demonstrates the potential of integrating active
learning into survey astronomy, however it does not 
account for the importance of
generating further 
observations about a source or the cost of these observations.  
The active learner implemented by
Richards {\it et al.} (2012a)
simply asks its human operators to provide classifications without offering
any guidance how.  Certainly, the humans would be almost guaranteed to
find the proper classification if they were to take both a full spectrum
and light curve of
the source, but this would come at the expense of a great deal of time and
effort.  We propose to eliminate this burden by using active learning not
only to designate objects to be added to the training set, but to designate
observations of those objects that would be the most efficient.  Figure 8 of
Richards {\it et al}. (2011) and Figure 13 of Bloom {\it et al}. (2011)
demonstrate that, for different classes of objects, different features are
more or less important.  If the machine learner already has an idea to what
class the queried object belongs, it should be able to give the operator a
recommendation of what observation to perform.  Maybe only a few color
measurments are needed to determine the object's physical origin.  
Maybe observations at only a few select epochs, rather than a drawn out time
series, will be needed to infer the object's complete light curve.
Maybe the required observations are consistent with the survey telescope's 
existing schedule, and no special follow-up will be required at all.  In this
way, active learning will allow us not only to maximize the science output of
our surveys, but to do so with the most efficient allocation of resources,
ensuring that we gain the most information about the most objects with the
least time and effort.

Active learning is integral to the identification and classification
of anomalies in cosmological data streams (e.g.\ through the
optimization of training sets used in classification).  Accomplishing this requires that we
expand the algorithms described previously to include active search,
and active feature acquisition. {\it Active feature acquisition}
entails deciding for each source, whether observations of additional
features (e.g.\ the colors of a source) would be valuable in
classifying an object.  Our recent work using Gaussian Process
(GP) regression shows that anomalous spectra can be inferred even 
from noisy input data (Garnett et al 2012). The key
advantage of this approach is that the GPs naturally provide a mean
and covariance for future unobserved parameters.  This uncertainty
propagates through to class labels and we can use it, for example, to
estimate the reduction in class uncertainty that will be gained by
observing a certain color at a given time.

In {\it active search} the ultimate objective in following up detected
anomalies is to maximize the number of interesting anomalies
classified and characterized while staying within a budget of
follow-up observations or experiments. The problem and the Bayesian
optimal algorithm for it are described in Garnett et al 2011 and Garnett
et al 2012a. We propose to expand on these approaches using scan
statistics to consider not just individual anomalies but also group
anomaly detection algorithms that consider arbitrary groupings of
self-similar anomalous records (Neil and Moore 2005).

Combining these approaches and adapting them to on-line or streaming
data (i.e.\ rather than considering an entire pool of test objects,
anomalies appear one at a time as they are detected and learning
algorithms must decide whether and how to follow up on each as they
are detected) we will address whether additional performance
improvements can be made through an integrated model and decision
algorithm. 

\noindent{\bf 4.\ Inference in the Presence of Noise and Gaps in a Data
  Stream:}
The final component of this research program will be the development
of a class of active learning methods that are robust to the presence
of noise and incomplete data. Even given the data volumes for the next
generation cosmological surveys their temporal sampling of the sky
will be poor (twice every three nights). Techniques such as GPs offer
two distinct advantages which render them particularly amenable to
integration into active learning frameworks.  Variance can be used
to assign an uncertainty to the predicted value and the covariance
provides the structure between all pairs of predicted and observed
outputs.  Bryan {et al} (2007) and Daniel {et al} (2012) use these
aspects to great affect, treating the determined variances as a
measure of the information that can be learned by promoting a query
point to a new training point and thus learning the likelihood surface
of a theory space with greater efficiency than traditional MCMC
methods.

The second advantage of Gaussian processes is their indifference to
the physical meaning of the inputs and outputs.  There is nothing
(other than computational cost) to prevent us from adding arbitrary
sets of variables and including them into the covariance
structure. This generalization makes them broadly applicable to a wide
range of experimental data. Indeed, Gaussian processes allow us to
incorporate any measured attribute (e.g.\ morphology, nearest-neighbor
distance) of our sources in a principled manner and use those
attributes to control the spread and bias in our estimation
techniques. In such a way, we can incorporate the measurement
uncertainties in our photometric colors and propagate them
consistently through to uncertainties in the determined photometric
redshifts. We propose to expand upon these directions to consider the impact
of not just measurement uncertainties but also missing data in terms
of characterizing the covariance.


