\documentclass[useAMS,usenatbib,tightenlines,11pt,preprint]{aastex}
%\documentclass[useAMS,usenatbib,tightenlines,11pt,preprint]{aastex}
\usepackage[paperwidth=8.5in,paperheight=11in,centering,margin=1in]{geometry}

\usepackage{parskip}
%\setlength{\parskip}{\baselineskip}
\parskip=5pt

\usepackage{amsmath}
\usepackage{amsbsy}

\input epsf
\usepackage{amsmath,amssymb,subfigure}
\usepackage{graphicx}
\usepackage{epsfig}
\usepackage{color}
%\usepackage{ulem}
%\usepackage{epstopdf}

\usepackage{multicol}
%\usepackage{etoolbox}

\pagestyle{empty}

\renewcommand{\baselinestretch}{0.99}

%%%%%%%%%%%%%%%%%%%%%%%%%%%%%%%%%%%%%%%%%%%%%%%%%%%%%%%%%%%%
%%%%%%%%%%%%%%%%%%%%%%%%%%%%%%%%%%%%%%%%%%%%%%%%%%%%%%%%%%%%
%%%%%%%%%%%%%%%%%%%%%%%%%%%%%%%%%%%%%%%%%%%%%%%%%%%%%%%%%%%%

\begin{document} 

\begin{center}
{\bf \Large Learning in an Era of Uncertainty}
\end{center}

\vspace{1cm}

\noindent
\begin{tabular}{ll}
{\bf Principal Investigator: } &Andrew Connolly \\
{\bf Applicant/institution: } & University of Washington\\
{\bf Telephone number: } & (206) 543 9541 \\
{\bf Email: } & ajc@astro.washington.edu \\
{\bf Funding request:} & \$170K (year one), \$170K (year two), \$170K (year
three) \\
{\bf DOE/OSP Office: } & Office of Advanced Scientific Computing Research \\
{\bf DOE/Technical Contact: } & Dr. Alexandra Landsberg 
\end{tabular}

\vspace{1cm}


\noindent \begin{tabular}{ll}
{\bf CoPrincipal Investigator: } & Jeff Schneider \\
{\bf Applicant/institution: } & Carnegie Mellon University \\
{\bf Telephone number: } & (412) 268 2339 \\
{\bf Email: } & schneide@cs.cmu.edu \\
{\bf Funding request:} & \$170K (year one), \$170K (year two), \$170K (year 
three) 
\end{tabular}
 
\vspace{1cm}

\noindent \begin{tabular}{ll}
{\bf Total Funding request:} & \$340K (year one), \$340K (year two), \$340K
(year  three)
\end{tabular}

% \label{firstpage}

% \maketitle 
\newpage

\begin{center}
{\bf \Large Learning in an Era of Uncertainty}
\end{center}

\section{Introduction:}
A new generation of DOE sponsored data intensive experiments and
surveys, designed to address fundamental questions in physics and
biology, will come on-line over the next decade. These experiments
share many similar challenges in the field of statistics and
machine-learning: how can we choose the next experiment or observation
to make in order that we maximize our scientific returns; how can we
identify anomalous sources (that may be indicative of new events or
potential systematics within our experiments) from a continuous stream
of data; how do we characterize and classify correlations and events
within data streams that are noisy and incomplete. This goal of this
proposal is to address these challenges through the development of a broad
class of novel and scalable machine-learning techniques centered
around the theme of active learning.

{\it Active learning} algorithms iteratively decide on which data
points they will collect outputs on and add to a training set.  Their
goal is to choose the points that will most improve the model being
learned.  At each step, they consider the current training data, the
potential training data that could be collected, and the current
learned model, and evaluate what would be the best choice for the next
observation, experiment, or feature such that it improve the our
knowledge of the system (according to some objective criterion).
% something more on their use and potential

% While the basic formalism for active learning is well understood
% there are many challenges associated with data streams...

While the algorithms and methodologies we propose will impact many of
the data intensive sciences, we will focus our work in the context of
DOE sponsored cosmology experiments (i.e.\ the Dark Energy Survey, and
the Large Synoptic Survey Telescope). These surveys are ideal proxies
as their bandwidth (terabytes of data per night and petabytes of data
every couple of months) will enable high precision studies impacting
the understanding of cosmology, particle physics, and potentially
theories of gravity.  Our ability to achieve these scientific goals
relies on analyses at a scale, speed, and complexity beyond the
capabilities of current automated machine learning methods.


\section{Active learning for calibrating cosmology}

Over the last decade a concordance model has emerged for the universe
that describes its energy content. The most significant contribution
to the energy budget today comes in the form of ``dark energy'', which
explains the observation that we reside in an accelerating
universe. Despite its importance to the formation and evolution of the
universe there is no compelling theory that explains the energy
density nor the properties of the dark energy. Understanding the
nature of dark energy remains as one of the most fundamental questions
in Physics today, impacting our understanding of cosmology, particle
physics, and potentially theories of gravity itself.  As noted in the draft
report of the Dark Energy Task Force (DETF; constituted jointly by
DOE, NSF and NASA), ``the nature of dark energy ranks among the very
most compelling of all outstanding problems in physical science''.


To address the question of the nature of dark energy a new generation
of DOE sponsored experiments are entering service (e.g.\ the Dark
Energy Survey, DES\footnote{http://www.darkenergysurvey.org} and the
Large Synoptic Sky Survey, LSST\footnote{http://www.lsst.org}).  These
surveys will represent a 40-fold increase in data rates over current
experiments (generating over 100 Petabytes of data over a period of 10
years) and decreasing the uncertainties on our measures of the
underlying properties of dark energy by more than a factor of ten. On
these scales, statistical noise will no longer determine the accuracy
to which we can measure cosmological parameters. The control and
correction of systematics will ultimately determine our final
figure-of-merit. Prime amongst these systematics is the estimation of
distances to extragalactic sources, the identification of anomalous
events within the temporal universe (e.g.\ detecting optical flashes
and supernovae at cosmological distances), and the real time
classification of data in the presence of uncertainties and gaps
within the data stream.

%the identification and characterization of novel
%sources requires sifting through millions of ``uninteresting'' objects
%to locate those truly worth studying.  The next generation of
%astronomy will be defined by large surveys such as
%LSST\footnote{http://www.lsst.org}, each of which promises to yield
%terabytes of data every night.  

%These surveys will pose challenges not
%only because of the volume of data they will provide, but the type, as
%their repeated observations of the same regions of sky will finally
%open detailed time-domain astronomy to physical scrutiny
%\cite{sciencebook}.  Given the billion dollar investments being made
%in these surveys, it is paramount that we develop techniques that can%
%optimize their science output.

\section{Inference in the Presence of Noise and Gaps in a Data
  Stream}




\subsection{Anomaly Detection and Classification in Massive Data
 Streams}

The next generation of astrophysical surveys we will visit the same
region of sky many thousands of time. This opening of the temporal
domain in astrophysics offers the potential to discover new classes of
physical phenomena while coming with many associated computational
challenges. Variability within the universe is believed to be present
on time scales of seconds through to tens of years. The shortest time
scales correspond to the explosion of the most massive stars within
the universe which produce short but intense optical and gamma-ray
flashes. These outbursts provide direct tests of General Relativity
and of high energy physical processes (at energies far beyond those
accessible on the Earth). For example the rate at which these events
occur constrains the age at which the first stars within the universe
came into being. Intermediate timescale variability comes in the form
of supernovae (SNe) which detonate, brighten and then dim. These
exploding stars are known to have a narrow range of intrinsic
brightnesses; they act as standard candles that can be used to
determine the rate at which the universe expands and thereby measure
its mass and energy content \cite{perlmutter99}. 
%Longer time scale
%variablity (including variability due to the velocity of sources)
%comes from variations in the luminosity of acretion disks around black
%holes, the motion of stars throughout our Galaxy and the motion of
%asteroids within the local solar system. These...
With surveys such as the LSST we will detect 250,000 SNe per year
increasing the accuracy of measures of the energy content of the
universe by an order of magnitude.

With timescales as short as seconds to hours we need to be able to
identify, classify and report any detection in time to allow for
followup observations before the initial outburst
fades. Identification and classification must, therefore, be
undertaken in almost real-time with probabilistic classifications that
incorporate our uncertainties about our classification together with
the ability for algorithms to learn based on a posteriori information
from earlier classifications. It must be able to predict what
additional information might be needed to improved (or exclude) the
likelihood of a given classification and to specify which parameters
led to the source being classified as anomalous. Small errors in the
identification and classification of these anomalous sources will
swamp any underlying signal. The LSST will detect 7.5x10$^8$ sources
{\bf every night}. Even for the most numerous transient events (SNe)
this corresponds to less than 10$^{-5}$ of the total number of sources
identified being transient. For the most energetic bursters the
magnitude of the challenge is 500-fold larger.  Algorithms for
identifying anomalies and variability must, therefore, be robust to
false positives and missing data and must account for the cadence in
how we sample the time domain, variations in the quality of the data
due to atmospheric conditions, changes in the performance of the
telescope and camera and the possibility that we observe sources at
different wavelengths at different times.


RR-Lyrae - figure showing sampling






\section{Experimental Design and Optimization through Active
  Learning}
 
Experimental design description

\subsection{Active learning for calibrating cosmology}

The accurate determination of an object's distance or redshift is
central to every test of cosmology that happens outside of a particle
accelerator.  The comparison of redshifts and luminosities of standard
candles enabled the discovery of dark energy and cosmic acceleration
\cite{perlmutter1998}. Redshifts serve as a proxy for radial distance
from Earth to the observed object.  Redshifts thus are necessary for
building three dimensional maps of the distribution of galaxies in the
Universe.  Such maps will help and have helped us to constrain how
galaxies formed over the history of the Universe, and thus can tell us
much about how gravity operates at the largest scales and what the
parameters are that govern the behavior of dark energy and the cosmic
acceleration \cite{?}.  Accurately determining the redshifts of
distant galaxies is a requirement if we are to answer some of the most
vexing problems in fundamental physics today.

Direct spectroscopic redshift measurements of enough galaxies to
constrain dark energy parameters to the precision required by next
generation experiments would be thousands of times more expensive than
taking the corresponding photometric (or imaging) data.  Large digital
cameras (e.g.\ the 3.2 Gigapixel camera for the LSST) can observe
$\sim 10^6$ sources every 15 seconds (several orders of magnitude more
efficient that spectroscopic observations). Our task, then, is to
construct algorithms whereby we can convert these much cheaper
photometric data into accurate redshifts (i.e. photometric redshifts).


The demands of next generation cosmological experiments will require
that our photometric redshift determinations be accurate to within
$\le 2\times 10^{-3}(1+z)$ \cite{desc}.  This is a hard limit, as a
bias in redshift determination of just 0.01 can degrade dark energy
constraints by as much as 50\%
\cite{kitching,huterer2006,nakajima2011}.  Testing present template
and empirical methods on a sample of 5,482 galaxies from the 2df-SDSS
LRG and Quasar survey, Abdalla {\it et al}. (2011) find biases of
order 0.05 (see their Figure 4).  This level of bias can degrade dark
energy constraints by as much as a factor of 3 \cite{Ma2006}.
Considering 3,000 galaxies from the DEEP2 EGS and zCOSMOS surveys and
using Bayesian methods, Mandelbaum {\it et al}.  (2008) find a bias in
redshift determination of order 0.01 (see their Table 2).  While this
is an improvement, it still an order of magnitude larger than what is
required.

How can we resolve these problems?  In both the empirical and template
photometric redshift codes, biases arise from the fact that the
training samples (templates) do not occupy the same color and redshift
space as the data.  Improving on this through targeted observations
is, however, expensive.  Simple sampling strategies (e.g. random or
stratefied) are not efficient.  We need a technique to identify the
next best observation to take to best reduce the redshift estimation
bias of our algorithm.  Active learning will provide this technique.

Focus on these as our primary test to develop these algorithms

\subsection{Active learning}



In the field of machine learning, {\it active learning} algorithms
iteratively decide which data points they will collect outputs on and
add to a training set.  The goal is to choose the points that will
most improve the model being learned.  At each step, they consider the
current training data, the potential training data that could be
collected, and the current learned model, and evalute each potential
new point according to some objective criterion.  They are
particularly valuable when the data in question are expensive to
acquire (in time or resources). Many examples of such problems exist
in physics and cosmology: from defining training sets for estimating
the distances to galaxies based on their photometric data (Connolly et
al 1995), to choosing which observation will most improve a
cosmological signal, to picking the next anomaly to follow-up, or the
next piece of data to obtain about an anomaly that would enable its
accurate classification.

%The computational and statistical challenges for searching for optimal
%classifications or sets of experiments are
%exacerbated by the usual desire to construct a model over the {\em entire}
%state space of a system.  For example, when learning classifiers, we
%typically score the performance based on prediction accuracy over an entire
%test set.  It is less common to offer the system a chance to browse the
%test set and report whatever interesting observations it can make about it.
%The problem becomes worse when we attempt to follow model learning with
%discovery.  Optimizing over learned models is always risky since the
%optimizer is likely to find the learner's mistakes.  


%Small mistakes in the learning process can lead to big mistakes
%in identifying anomalies.  For example, suppose a structure learning
%algorithm fails to include a link between two variables that are
%highly related at least for a subset of their possible values.  All
%records that are anomalies based on an unlikely combination of those
%attributes will be missed.  Similarly, a spurious additional link with
%poorly learned parameters will create anomalies where there are none.
%Mistakes like these are inevitable in large models.

A classic active learning method, called uncertainty sampling, uses
the uncertainty of each test point as the criterion for choosing the
next experiment (e.g.\ the next spectroscopic measurement of a source
in the training sample).  We will expand upon these approaches using
our recent work on the problem of optimal surveying or polling
(Garnett et al 2012a).  Rather than having a goal of correctly
predicting the output for each point in a test set, the goal is to
predict the average output (or the class proportions in classification
problems) over the test set.  This dramatically increases the
efficiency of the active learning.  In preliminary experiments on
graphs and other domains, minimizing this survey variance not only
performs well on the surveying problem, but also outperforms the trace
criterion and other popular active learning methods such as
uncertainty and density sampling on active learning problems.
Intuitively, it seems reasonable that considering the entire
covariance matrix might lead to better performance than choosing only
based on its diagonal.  We have, however, little theoretical understanding of
why this is better than the trace criterion which directly optimizes
the quantity on which we will ultimately measure performance.  We will
seek a better theoretical understanding of this phenomenon as part of
this work.



Information gain and astronomy (Jake's stuff)








\end{document}

\subsection{Active Learning and Photometric Redshifts}
\label{sec:mlpz}

The discussion above illustrates how the appropriate choice of machine
learning algorithm can affect the fidelity of one's photometric redshift
determinations.  Further gains can be made if one is similarly judicious in
choosing a training set.  While next generation surveys like the LSST will
be exclusively photometric, they will present us with large numbers of
galaxies which we will have the option to follow-up with off-site
spectroscopy.  It behooves us, therefore, to develop a quantitative way of
determining which galaxies represent the most effective use of these
limited follow-up resources.  Such methods fall in the field of active
learning and we will develop novel algorithms for this purpose.

Active learning algorithms iteratively decide which data points they will
collect outputs on and add to the training set.  The goal is to choose the
points that will most improve the model being learned.  At each step, they
consider the current training data, the potential training data that could
be collected, and the current learned model, and evalute each potential new
point according to some objective criterion.  After selecting one or more
new points, the outputs for those points are collected and added to the
training set and the process repeats.  The key to the active learning
algorithm is the specification of the objective criterion used for
selection.  Settles (2009) presents a survey of such criteria.



Real-time automatic classification of objects is already widely acknowledged as a
necessary support technology for the forthcoming age of survey astronomy
\cite{djorgovski2011,richards2011,richards2012,graham2012,mahabal2008a,mahabal2011a}.
Objects will need to be categorized into known science classes so that novel
or rare objects can be flagged for detailed follow-up observations.
For transient events, algorithms must be able to make rapid
decisions so that sources can be targeted for follow-up 
and classifications learned before
objects return to their quiescent phases.  A great deal of work has
already been done developing algorithms that can learn the classification of an
object given a fixed set of observations and training data.  
Mahabal {\it et al}. (2008a,2011a,2011b) propose to break down the
observations of a given object into $\{\Delta m,\Delta t\}$ pairs (where $m$
is magnitude and $t$ is time) and use the density of observations in this 
two-dimensional space as the basis for a Bayesian classification algorithm.
Mahabal {\it et al}. (2008b) alternatively propose to use those same 
$\{\Delta m,\Delta t\}$ pairs as the input to a Gaussian process regression
by which they will reconstruct the object's entire light curve as a function
of time, and then classify the object based on that reconstruction.
Richards {\it et al}. (2011) use observations of transient objects to extract
periodic (e.g. the amplitude and frequency of the first two Fourier modes of
the object's light curve) and non-periodic (e.g. the variance and skewness of
all of the magnitude observations taken, regardless of their separation in
time) and feed those features into several tree-based classifiers.
They find misclassification rates lower than 30\% with their best method yielding a
misclassification rate of 22.8\%.  Using only non-periodic features, which will be
especially easy for survey telescopes to gather, rather than
full light curves, they 
find a misclassification rate of
between 26\% and 28\%.
Bloom {\it et al}. (2011) also use a tree-based automatic classifier on
Palomar Transient Factory data and find a 3.8\% error rate when
discriminating between four major classifications.  Richards {\it et al}.
consider a more complete set of 25 possible classifications.
Clearly, many possibile approaches are available for the automated
classification of transient objects, and not all of them rely upon highly
detailed observations to function.

While significant attention has been paid to the problem of classifying an
object once training sets and observations have been assembled, relatively
little has been paid to optimizing the training set or observations.
Though the budgeted observing programs of survey telescopes leave little
time to follow-up serendipitous discoveries, other telescopes will be much
freer to fill in the gaps in survey-derived knowledge.  The question is
how best to do so.  The problem of sub-optimal training data is already being
faced by the community.  Figure 6 of Richards {\it et al}. (2011)
demonstrates that automated classifiers are much better at identifying
objects of common types, for which training data is abundant, than they are
at identifying objects of rare types, for which training data is sparse.
The authors address this problem in Richards {\it et al.} (2012) by designing
an active learning algorithm which selectively adds to the training set,
choosing objects of uncertain classification and asking human classifiers to
provide detailed information (via either their expertise alone or with
follow-up observations) about the queried object.  They find that this scheme
reduces the misclassification rate 
of their classifier by approximately five percentage
points over a scheme which randomly constructs its training data set (see
their Figure 6).

This work demonstrates the potential of integrating active
learning into survey astronomy, however it does not 
account for the importance of
generating further 
observations about a source or the cost of these observations.  
The active learner implemented by
Richards {\it et al.} (2012)
simply asks its human operators to provide classifications without offering
any guidance how.  Certainly, the humans would be almost guaranteed to
find the proper classification if they were to take both a full spectrum
and light curve of
the source, but this would come at the expense of a great deal of time and
effort.  We propose to eliminate this burden by using active learning not
only to designate objects to be added to the training set, but to designate
observations of those objects that would be the most efficient.  Figure 8 of
Richards {\it et al}. (2011) and Figure 13 of Bloom {\it et al}. (2011)
demonstrate that, for different classes of objects, different features are
more or less important.  If the machine learner already has an idea to what
class the queried object belongs, it should be able to give the operator a
recommendation of what observation to perform.  Maybe only a few color
measurments are needed to determine the object's physical origin.  
Maybe observations at only a few select epochs, rather than a drawn out time
series, will be needed to infer the object's complete light curve.
Maybe the required observations are consistent with the survey telescope's 
existing schedule, and no special follow-up will be required at all.  In this
way, active learning will allow us not only to maximize the science output of
our surveys, but to do so with the most efficient allocation of resources,
ensuring that we gain the most information about the most objects with the
least time and effort.

Active learning is integral to the identification and classification
of anomalies in cosmological data streams (e.g.\ through the
optimization of training sets used in classification).  Accomplishing this requires that we
expand the algorithms described previously to include active search,
and active feature acquisition. {\it Active feature acquisition}
entails deciding for each source, whether observations of additional
features (e.g.\ the colors of a source) would be valuable in
classifying an object.  Our recent work using Gaussian Process
(GP) regression shows that anomalous spectra can be inferred even 
from noisy input data (Garnett et al 2012). The key
advantage of this approach is that the GPs naturally provide a mean
and covariance for future unobserved parameters.  This uncertainty
propagates through to class labels and we can use it, for example, to
estimate the reduction in class uncertainty that will be gained by
observing a certain color at a given time.

In {\it active search} the ultimate objective in following up detected
anomalies is to maximize the number of interesting anomalies
classified and characterized while staying within a budget of
follow-up observations or experiments. The problem and the Bayesian
optimal algorithm for it are described in Garnett et al 2011 and Garnett
et al 2012a. We propose to expand on these approaches using scan
statistics to consider not just individual anomalies but also group
anomaly detection algorithms that consider arbitrary groupings of
self-similar anomalous records (Neil and Moore 2005).

Combining these approaches and adapting them to on-line or streaming
data (i.e.\ rather than considering an entire pool of test objects,
anomalies appear one at a time as they are detected and learning
algorithms must decide whether and how to follow up on each as they
are detected) we will address whether additional performance
improvements can be made through an integrated model and decision
algorithm. 

\noindent{\bf 4.\ Inference in the Presence of Noise and Gaps in a Data
  Stream:}
The final component of this research program will be the development
of a class of active learning methods that are robust to the presence
of noise and incomplete data. Even given the data volumes for the next
generation cosmological surveys their temporal sampling of the sky
will be poor (twice every three nights). Techniques such as GPs offer
two distinct advantages which render them particularly amenable to
integration into active learning frameworks.  Variance can be used
to assign an uncertainty to the predicted value and the covariance
provides the structure between all pairs of predicted and observed
outputs.  Bryan {et al} (2007) and Daniel {et al} (2012) use these
aspects to great affect, treating the determined variances as a
measure of the information that can be learned by promoting a query
point to a new training point and thus learning the likelihood surface
of a theory space with greater efficiency than traditional MCMC
methods.

The second advantage of Gaussian processes is their indifference to
the physical meaning of the inputs and outputs.  There is nothing
(other than computational cost) to prevent us from adding arbitrary
sets of variables and including them into the covariance
structure. This generalization makes them broadly applicable to a wide
range of experimental data. Indeed, Gaussian processes allow us to
incorporate any measured attribute (e.g.\ morphology, nearest-neighbor
distance) of our sources in a principled manner and use those
attributes to control the spread and bias in our estimation
techniques. In such a way, we can incorporate the measurement
uncertainties in our photometric colors and propagate them
consistently through to uncertainties in the determined photometric
redshifts. We propose to expand upon these directions to consider the impact
of not just measurement uncertainties but also missing data in terms
of characterizing the covariance.

\begin{thebibliography}{99}

\bibitem[Abazajian {\it et al}. 2009]{Abazajian:2008wr}
  Abazajian, K.~N.{\it et al.}  [SDSS Collaboration]~2009,
  %``The Seventh Data Release of the Sloan Digital Sky Survey,''
  The Astrophysical Journal Supplement Series  {\bf 182}, 543
  [arXiv:0812.0649 [astro-ph]].
  %%CITATION = APJSA,182,543;%%

\bibitem[Abdalla {\it et al}. 2011]{abdalla}
Abdalla,~F.~B., Banerji,~M., Lahav,~O., and Rashkov,~V. 2011,
Monthly Notices of the Royal Astronomical Society {\bf 417}, 1891

\bibitem[Abramo {\it et al}. 2012]{narrow}
Abramo,~L.~R., Strauss,~M.~A., Lima,~M., Hern\'andez-Monteagudo,~C., Lazkoz,~R.,
Moles,~M., de Oliveira,~C.~M., Sendra,~I., Sodr\'e Jr.,~L., and
Storchi-Bergmann,~T. 2012, Monthly Notices of the Royal Astronomical Society
{\bf 423}, 3251

\bibitem[Albrecht {\it et al}. 2006]{detf}
Albrecht,~A., Bernstein,~B., Cahn,~R., Freedman,~W.~L., Hewitt,~J.,
Hu,~W., Huth,~J., Kamionkowski,~M., Kolb,~E., Knox,~L., Mather,~J.~C.,
Staggs,~S., Suntzeff,~N.~B. (Dark Energy Task Force) 2006,
``Report of the Dark Energy Task Force,''
\verb|http://jdem.gsfc.nasa.gov/science/DETF_Report.pdf|


\bibitem[Berg\'e {\it et al}. 2012]{psf}
Berg\'e,~J., Price,~S., Amara,~A., and Rhodes,~J. 2012,
Monthly Notices of the Royal Astronomical Society {\bf 419}, 2356

\bibitem[Bloom {\it et al}. 2011]{bloom2011}
Bloom,~J.~S., Richards,~J.~W., Nugent,~P.~E., Quimby,~R.~M., Kasliwal,~M.~M.,
Starr,~D.~L., Posnanski,~D., Ofek,~E.~O., Cenko,~S.~B., Butler,~N.~R.,
Kulkarni,~S.~R., Gal-Yam,~A., and Law,~N. 2011 [arXiv:1106.5491]

\bibitem[Bonfield {\it et al}. 2010]{bonfield}
Bonfield,~D.~G., Sun,~Y., Davey,~N., Jarvis,~M.~J., Abdalla,~F.~B.,
Banerji,~M., Adams,~R.~G. 2010, Monthly Notices of the Royal Astronomical Society 
{\bf 405} 987

\bibitem[Brammer {\it et al}. 2008]{eazy}
Brammer,~G.~B., van Dokkum,~P.~G., and Coppi,~P. 2008,
The Astrophysical Journal {\bf 686}, 1503

\bibitem[Bryan 2007]{brentsthesis}
Bryan, B., 2007, Ph.D. thesis
\verb|http://reports-archive.adm.cs.cmu.edu/anon/|
\verb|ml2007/abstracts/07-122.html|

\bibitem[Bryan {\it et al}. 2007]{bryan}
Bryan, B., Schneider, J., Miller, C.~J., Nichol, R.~C., Genovese, C., and
Wasserman, L., 2007,
The Astrophysical Journal {\bf 665}, 25

\bibitem[Budav\'ari 2008]{budavari2008}
Budav\'ari,~T. 2008 The Astrophysical Journal {\bf 695}, 747

\bibitem[Collister and Lahav 2004]{annz}
Collister,~A.~A. and Lahav,~O. 2004,
Publications of the Astronomical Society of the Pacific {\bf 116}, 345

\bibitem[Connolly {\it et al}. 2005]{imsim}
Connolly,~A.~J., Peterson,~J., Jernigan,~J.~G., Abel,~R., Bankert,~J.,
Chang,~C., Claver,~C.~F., Gibson,~R., Gilmore,~D.~K., Grace,~E., Jones,~R.~L.,
Ivezic,~Z., Jee,~J., Juric,~M., Kahn,~S.~M., Krabbendam,~V.~L., Krughoff,~S.,
Lorenz,~S., Pizagno,~J., Rasmussen,~A., Todd,~N. Tyson,~J.~A., and Young,~M.
2005, Society of the Photo-Optical Instrumentation Engineers (SPIE) Converence
Series {\bf 7738}, 53

\bibitem[Cunha {\it et al}. 2012]{cunha2012}
Cunha,~C.~E., Huterer,~D., Lin,~H., Busha,~M.~T., and Wechsler,~R.~H. 2012,
[arXiv:1207.3347]

\bibitem[Daniel and Linder 2010]{muvarpi2}
Daniel,~S.~F. and Linder,~E.~V. 2010, Physical Review D {\bf 82}, 103523

%\cite{Daniel:2011rr}
\bibitem[Daniel {\it et al}. 2011]{daniel2011} 
  Daniel,~S.~F., Connolly,~A.~J., Schneider,~J., Vanderplas,~J. and Xiong,~L. 
  %``Classification of Stellar Spectra with LLE,''
  The Astronomical Journal  {\bf 142}, 203 (2011)
  [arXiv:1110.4646 [astro-ph.SR]].
  %%CITATION = ARXIV:1110.4646;%%

\bibitem[Daniel {\it et al}. 2012]{daniel2012}
Daniel,~S.~F., Connolly,~A.~J., and Schneider,~J. 2012
[arXiv:1205.2708]

\bibitem[Das {\it et al}. 2011]{sudeep}
Das,~S., de Putter,~R., Linder,~E.~V., and Nakajima,~R. 2011,
[arXiv:1102.5090]

\bibitem[Davis {\it et al}. 2007]{essence}
Davis,~T.~M., M\"ortsell,~E., Sollerman,~J., Becker,~A.~C., Blondin,~S.,
Challis,~P., Clocchiatti,~A., Filippenko,~A.~V., Foley,~R.~J., Garnavich,~P.~M.,
Jha,~S., Krisciunas,~K., Kirshner,~R.~P., Leibundgut,~B., Li,~W., Matheson,~T.,
Miknaitis,~G., Pignata,~G., Rest,~A., Riess,~A.~G., Schmidt,~B.~P.,
Smith,~R.~C., Spyromilio,~J., Stubbs,~C.~W., Suntzeff,~N.~B., Tonry,~J.~L.,
Wood-Vasey,~W.~M., and Zenteno,~A. 2007, The Astrophysical Journal, {\bf 666},
716

\bibitem[de Putter {\it et al}. 2010]{roland}
de Putter,~R., Huterer,~D. and Linder,~E.~V. 2010, Physical Review D {\bf 81},
103513


\bibitem[Djorgovski {\it et al}. 2011]{djorgovski2011}
Djorgovski,~S.~J., Donalek,~C., Mahabal,~A.~A., Moghaddam,~B., Turmon,~M.,
Graham,~M.~J., Drake,~A.~J., Sharma,~N. and Chen,~Y. 2011
[arXiv:1110.4655] to appear in Statistical Analysis and Data Mining, ref. proc.
CIDU 2011 conf., eds. A. Srivastava and N. Chawla

\bibitem[Garnett {\it et al}. 2011]{Garnett11}
Garnett,~R., Krishnamurhty,~Y., Wang,~D., Schneider,~J., and Mann,~R. 2011,
``Bayesian Optimal Active Search on Graphs,'' KDD Workshop on Mining and
Learning with Graphs

\bibitem[Garnett {\it et al}. 2012a]{Garnett12}
Garnett,~R., Krishnamurthy,~Y., Xiong,~X., Schneider,~J., and Mann,~R. 2012a,
``Bayesian Optimal Active Search and Surveying,'' International Conference on
Machine Learning

\bibitem[Garnett {\it et al}. 2012b]{Garnett12a}
Garnett,~R., Ho,~S., and Schneider,~J. 2012b,
``Gaussian Processes for Identifying Damped Lyman-alpha Systems in Spectroscopic
Surveys,'' Neural Information Processing Systems 
workshop on Modern Nonparametric Methods in Machine Learning

\bibitem[Graham {\it et al}. 2012]{graham2012}
Graham,~M.~J., Djorgovski,~S.~G., Mahabal,~A., Donalek,~C., Drake,~A.,
Longo,~G. 2012 [arXiv:1208.2480] to appear in special issue of Distributed and
Parallel Databases on Data Intensive eScience

\bibitem[Huterer {\it et al}. 2006]{huterer2006}
Huterer,~D., Takada,~M., Bernstein,~G., and Jain,~B. 2006,
Monthly Notices of the Royal Astronomical Society {\bf 366}, 101

\bibitem[Kaufman {\it et al}. 2011]{kaufman}
Kaufman, Cari G., Bingham, Derek, Habib, Salman, Heitmann, Katrin, and Frieman,
Joshua A. 2011, The Annals of Applied Statistitcs {\bf 5} 2470

\bibitem[Kitching {\it et al}. 2008]{kitching}
Kitching,~T.~D., Taylor,~A.~N., and Heavens,~A.~F. 2008,
Monthly Notices of the Royal Astronomical Society {\bf 389} 173

\bibitem[Long {\it et al}. 2012]{long2012}
Long,~J.~P., El Karoui,~N., Rice,~J.~A., Richards,~J.~W., and Bloom,~J.~S. 2012,
Publications of the Astronomical Society of the Pacific {\bf 124} 280

\bibitem[LSST Collaboration 2011]{lsstoverview}
LSST Collaboration 2011, [arXiv:0805.2366]
\verb|http://www.lsst.org/lsst/overview/|

\bibitem[LSST Dark Energy Science Collaboration 2012]{desc}
LSST Dark Energy Science Collaboration 2012, [arXiv:1211.0310]

\bibitem[LSST Science Collaborations 2009]{sciencebook}
LSST Science Collaborations 2009, ``LSST Science Book'',
\verb|http://www.lsst.org/lsst/science/scibook|

\bibitem[Ma {\it et al}. 2006]{Ma2006}
Ma,~Z., Hu,~H., and Huterer,~D. 2006, The Astrophysical Journal {\bf 636}, 21

\bibitem[Ma {\it et al}. 2012]{YifeiMa12}
Ma,~Y., Garnett,~R., and Schneider,~J. 2012,
``Submodularity in Batch Active Learning and Survey Problems
on Gaussian Random Fields,''
Neural Information Processing Systems 
workshop on Discrete Optimization in Machine Learning

\bibitem[Mahabal {\it et al}. 2008a]{mahabal2008a}
Mahabal,~A., Djorgovski,~S.~G., Turmon,~M., Jewell,~J., Williams,~R.~R.,
Drake,~A.~J., Graham,~M.~G., Donalek,~C., Glikman,~E., and the Palomar-QUEST Team
2008a, Astronomische Nachrichten {\bf 329}, 288

\bibitem[Mahabal {\it et al}. 2008b]{mahabal2008b}
Mahabal,~A., Djorgovski,~S.~G., Williams,~R., Drake,~A., Donalek,~C.,
Graham,~M., Moghaddam,~B., Turmon,~M., Jewell,~J., Khosla,~A., and
Hensley,~B. 2008b [arXiv:0810.4527] to appear in proceedings fo the Class2008
conference (Classification and Discovery in Large Astronomical Surveys, Ringberg
Castle, 14-17 October 2008)

\bibitem[Mahabal {\it et al}. 2011a]{mahabal2011a}
Mahabal,~A.~A., Donalek,~C., Djorgovski,~S.~J., Drake,~A.~J.,
Graham,~M.~J., Williams,~R., Chen,~Y., Moghaddam,~B., and Turmon,~M.
2011a, [arxiv:1111.3699] to appear in Proc. IAU 285, ``New Horizons in Transient
Astronomy,'' Oxford, September 2011

\bibitem[Mahabal {\it et al}. 2011b]{mahabal2011b}
Mahabal,~A.~A., Djorgovski,~S.~G., Drake,~A.~J., Donalek~C., Graham,~M~J.,
Williams,~R.~D., Chen,~Y., Moghaddam,~B., Turmon,~M., Beshore,~E., and Larson,~S.
2011b, Bulletin of the Astronomical Society of India {\bf 39}, 387

\bibitem[Mandelbaum {\it et al}. 2008]{mandelbaum2008}
Mandelbaum,~R., Seljak,~U., Hirata,~C.~M., Bardelli,~S., Bolzonella,!M.,
Bongiorno,~A., Carollo,~M., Contini,~T., Cunha,~C.~E., Garilli,~B.,
Iovino,~A., Kambczyk,~P, Kneib,~J.-P., Knobel,~C., Koo,~D.~C., Lamareille,~F.,
Le F\`evre,~O., Leborgne,~J.-F., Lilly,~S.~J., Maier,~C., Mainieri,~V.,
Mignoli,~M., Newman,~J.~A., Oesch,~P.~A., Perez-Montero,~E., Ricciardelli,~E.,
Scodeggio,~M., Silverman,~J., and Tasca,~L. 2008, Monthly Notices of the Royal
Astronomical Society {\bf 386}, 781

\bibitem[McBride {\it et al}. 2011a]{mcbride2011a}
McBride,~C.~K., Connolly,~A.~J., Gardner,~J.~P., Scranton,~R., Newman,~J.~A.,
Scoccimarro,~R., Zehavi,~I., and Schneider,~D.~P. 2011a, The Astrophysical
Journal, {\bf 726}, 13

\bibitem[McBride {\it et al}. 2011b]{mcbride2011b}
McBride,~C.~K., Connolly,~A.~J., Gardner,~J.~P., Scranton,~R., Scoccimarro,~R.,
Berlind,~A.~A., Mar\'in,~F., and Schneider,~D.~P. 2011b, The Astrophysical
Journal {\bf 739}, 85

\bibitem[Moore {\it et al}. 2000]{Moore00}
Moore,~A., Connolly,~A., Genovese,~C., Grone,~L., Kanidoris,~N., Nichol,~R.,
Schneider,~J., Szalay,~A., Szapudi,~I., and Wasserman,~L. 2000,
``Fast Algorithms and Efficient Statistics: N-point Correlation Functions,'' in
MPA/MPE/ESO Conference on Mining the Sky [arXiv:astro-ph/0012333]

\bibitem[Nakajima {\it et al}. 2012]{nakajima2011}
Nakajima,~R., Mandelbaum,~R., Seljak,~U., Cohn,~J.~D., Reyes,~R., and
Cool,~R. 2012, Monthly Notices of the Royal Astronomical Society {\bf 420}, 3240
[arXiv:1107.1395]

\bibitem[Nichol {\it et al}. 2006]{Nichol2006}
Nichol,~R.~C., Sheth,~R.~K., Suto,~Y., Gray,~A.~J., Kayo,~I., Wechsler,~R.~H.,
Marin,~F., Kulkarni,~G., Blanton,~M., Connolly,~A.~J., Gardner,~J.~P., Jain,~B.,
Miller,~C.~J., Moore,~A.~W., Pope,~A., Pun,~J., Schneider,~D., Schneider,~J.,
Szalay,~A., Szapudi,~I., Zehavi,~I., Bahcall,~N.~A., Csabai,~I., Brinkmann,~J.
2006, Monthly Notices of the Royal Astronomical Society {\bf 368}, 1507

\bibitem[Poczos and Schneider 2011]{poczos11alphadiv}
Poczos,~B. and Schneider,~J. 2011, ``On the Estimation of alpha-Divergences,''
Artificial Intelligence and Statistics (AISTATS)

\bibitem[Poczos {\it et al}. 2011]{Poczos2011UAI}
Poczos,~B., Xiong,~L., and Schneider,~J. 2011, ``Nonparametric Divergence Estimation with
Applications to Machine Learning on Distributions,''  Uncertainty in Artificial
Intelligence

\bibitem[Poczos {\it et al}. 2012]{poczos12CVPR}
Poczos,~B., Xiong,~L., Sutherland,~D., and Schneider,~J. 2012,
``Nonparametric Kernel Estimators for Image Classification,''
IEEE Conference on Computer Vision and Pattern Recognition

\bibitem[Rasmussen and Williams 2006]{gp}
Rasmussen, C.~E. and Williams, C.~K.~I., 2006, ``Gaussian
Processes for Machine Learning''
\verb|http://www.GaussianProcess.org/gpml/|

\bibitem[Richards {\it et al}. 2004]{qso}
Richards,~G.~T., Nichols,~R.~C., Gray,~A.~G., Brunner,~R.~J., Lupton,~R.~H.,
Vanden Berk,~D.~E., Chong,~S.~S., Weinstein,~M.~A., Schneider,~D.~P.,
Anderson,~S.~F., Munn,~J.~A., Harris,~H.~C., Strauss,~M.~A., Fan,~X.,
Gunn,~J.~E., Ivezi\'c,~Z., York,~D.~G., Brinkmann,~J., and Moore,~A.~W. 2004,
The Astrophysical Journal Supplement Series, {\bf 155}, 257

\bibitem[Richards {\it et al}. 2011]{richards2011}
Richards,~J.~W., Starr,~D.~L., Butler,~N.~R., Bloom,~J.~S., Brewer,~J.~M.,
Crellin-Quick,~A., Higgins,~J., Kennedy,~R., and Rischard,~M. 2011,
The Astrophysical Journal {\bf 733}, 10

\bibitem[Richards {\it et al}. 2012]{richards2012}
Richards,~J.~W., Starr,~D.~L., Brink,~H., Miller,~A.~A., Bloom,~J.~S.,
Butler,~N.~R., James,~J.~B., Long,~J.~P., and Rice,~J. 2012
The Astrophysical Journal {\bf 744}, 192

\bibitem[Rosenfield {\it et al}. 2011]{rosenfield2011}
Rosenfield,~P., Connolly,~A., Fay,~J., Sayres,~C., and Tofflemire,~B. 2011,
Astronomical Society of the Pacific Conference Series {\bf 443}, 109

\bibitem[Scranton {\it et al}. 2002]{Scranton2002}
Scranton,~R., Johnston,~D., Dodelson,~S., Frieman,~J.~A., Connolly,~A.,
Eisenstein,~D.~J., Gunn,~J.~E., Hui,~L., Jain,~B., Kent,~S., Loveday,~J.,
Narayanan,~V., Nichol,~R.~C., O'Connell,~L., Soccimarro,~R., Sheth,~R.~K.,
Stebbins,~A., Strauss,~M.~A., Szalay,~A.~S., Sapudi,~I., Tegmark,~M.,
Vogeley,~M., Zehavi,~I., Annis,~J., Bahcall,~N.~A., Brinkman,~J., Csabai,~I.,
Hindsley,~R., Ivezic,~Z., Kim,~R.~S.~J., Knapp,~G.~R., Lamb,~D.~Q., Lee,~B.~C.,
Lupton,~R.~H., McKay,~T., Munn,~J., Peoples,~J., Pier,~J., Richards,~G.~T.,
Rockosi,~C., Schlegel,~D., Schneider,~D.~P., Stoughton,~C., Tucker,~D.~L.,
Yanny,~B., York,~D.~G. 2002, The Astrophysical Journal {\bf 579}, 48

\bibitem[Sesar {\it et al}. 2011]{linear}
Sesar,~B., Stuart,~J.~S., Ivezi\'c,~\u Z., Morgan,~D.~P., Becker,~A.~C., and
Wo\'zniak,~P. 2011, The Astronomical Journal {\bf 142}, 190

\bibitem[Settles 2009]{activelearning}
Settles,~B. 2009, ``Active Learning Literature Survey,'' Computer Sciences Technical
Report 1648, University of Wisconsin-Madison,
\verb|http://pages.cs.wisc.edu/~bsettles/active-learning/|


\bibitem[Shafieloo {\it et al}. 2012]{ericgp}
Shafieloo,~A., Kim,~A.~G., and Linder,~E.~V. 2012,
Physical Review D {\bf 85}, 123530 [arXiv:1204.2272]


\bibitem[Skibba {\it et al}. 2006]{Skibba2006}
Skibba,~R., Sheth,~R.~K., Connolly,~A.~J., and Scranton,~R. 2006,
Monthly Notices of the Royal Astronomical Society, {\bf 369}, 68

\bibitem[Straf 2003]{straf03}
Straf,~M.~L. 2003, Journal of the American Statistical Association {\bf 98}, 1

\bibitem[Szapudi {\it et al}. 2002]{Szapudi2002}
Szapud,~I., Frieman,~J.~A., Scoccimarro,~R., Szalay,~A.~S., Connolly,~A.~J.,
Dodelson,~S., Eisenstein,~D.~J., Gunn,~J.~E., Johnston,~D., Kent,~S.,
Loveday,~J., Meiksin,~A., Nichol,~R.~C., Scranton,~R., Stebbins,~A.,
Vogeley,~M.~S., Annis,~J., Bahcall,~N.~A., Brinkman,~J., Csabai,~I., Doi,~M.,
Fukigita,~M., Ivezi\'c,~\u Z., Kim,~R.~S.~J., Knapp,~G.~R., Lamb,~D.~Q.,
Lee,~B.~C., Lupton,~R.~H., McKay,~T.~A., Munn,~J., Peoples,~J., Pier,~J.,
Rockosi,~C., Schlegel,~D., Stoughtfon,~C., Tucker,~D.~L., Yanny,~B., York,~D.~G.
2002, The Astrophysical Journal {\bf 570}, 75

\bibitem[Vanderplas and Connolly 2009]{vdp2009}
Vanderplas,~J. and Connolly,~A.~J. 2009,
Astronomical Journal {\bf 138}, 1365

\bibitem[Wiley {\it et al}. 2011]{wiley2011}
Wiley,~K, Connolly,~A.~J., Gardner,~J., Krughoff,~S., Balazinska,~M., Howe,~B.,
Kwon,~Y., and Bu, ~Y. 2011, Publication of the Astronomical Society of the
Pacific {\bf 123}, 366

\bibitem[Xiong {\it et al}. 2011a]{Xiong2011gad}
Xiong,~L., Poczos,~B., Schneider,~J., Connolly,~A., Vanderplas,~J. 2011a,
``Hierarchical Probabilistic Models for Group Anomaly Detection,''
Artificial Intelligence and Statistics (AISTATS)

\bibitem[Xiong {\it et al}. 2011b]{xiong2011fgm}
Xiong,~L., Poczos,~B., and Schneider,~J. 2011, ``Group Anomaly Detection using Flexible
Genre Models,'' Neural Information Processing Systems

\bibitem[Yip {\it et al}. 2004]{yip2004a}
Yip,~C.~W., Connolly,~A.~J., Szalay,~A.~S., Budav\'ari,~T., SubbaRao,~M.,
Frieman,~J.~A., Nichol,~R.~C., Hopkins,~A.~M., York,~D.~G., Okamura,~S.,
Brinkmann,~J., Csabai,~I., Thakar,~A.~R., Fukugita,~M., 
and Ivezi\'c,~\u Z. 2004, The Astronomical Journal {\bf 128}, 585

\bibitem[Zhang and Schneider 2010a]{YiZhangICML2010}
Zhang,~Y. and Schneider,~J. 2010a, ``Projection Penalties: Dimension Reduction without
Loss,'' International Conference on Machine Learning

\bibitem[Zhang and Schneider 2010b]{YiZhangMultitask2010}
Zhang,~Y. and Schneider,~J. 2010b,
``Learning Multiple Tasks with a Sparse Matrix-Normal Penalty,''
Neural Information Processing Systems

\bibitem[Zhang {\it et al}. 2010]{YiZhangSDM2010}
Zhang,~Y., Schneider,~J., and Dubrawski,~A. 2010,
``Learning Compressible Models,'' Proceedings of SIAM Data Mining Conference

\bibitem[Zhang and Schneider 2011]{YiZhang2011multiECOC}
Zhang,~Y. and Schneider,~J 2011, ``Multi-label Output Codes using Canonical Correlation
Analysis,'' Artificial Intelligence and Statistics

\bibitem[Zhang and Schneider 2012]{YiZhang2012}
Zhang,~Y. and Schneider,~J. 2012, ``Maximum Margin Output Coding,''
International Conference on Machine Learning

\end{thebibliography} 
\label{lastpage}

\end{document}
