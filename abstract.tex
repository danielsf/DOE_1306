\documentclass[useAMS,usenatbib,tightenlines,11pt,preprint]{aastex}
%\documentclass[useAMS,usenatbib,tightenlines,11pt,preprint]{aastex}
\usepackage[paperwidth=8.5in,paperheight=11in,centering,margin=1in]{geometry}

\usepackage{parskip}
%\setlength{\parskip}{\baselineskip}
\parskip=5pt

\usepackage{amsmath}
\usepackage{amsbsy}

\input epsf
\usepackage{amsmath,amssymb,subfigure}
\usepackage{graphicx}
\usepackage{epsfig}
\usepackage{color}
%\usepackage{ulem}
%\usepackage{epstopdf}

\usepackage{multicol}
%\usepackage{etoolbox}

\pagestyle{empty}

\renewcommand{\baselinestretch}{0.99}

%%%%%%%%%%%%%%%%%%%%%%%%%%%%%%%%%%%%%%%%%%%%%%%%%%%%%%%%%%%%
%%%%%%%%%%%%%%%%%%%%%%%%%%%%%%%%%%%%%%%%%%%%%%%%%%%%%%%%%%%%
%%%%%%%%%%%%%%%%%%%%%%%%%%%%%%%%%%%%%%%%%%%%%%%%%%%%%%%%%%%%

\begin{document} 

\noindent {\bf Project Title: }  Learning in an Era of Uncertainty \\
{\bf Principal Investigator: } Andrew Connolly (University of Washington)\\
{\bf Major Participants: }  Jeff Schneider (Carnegie Mellon University) \\

\noindent{\bf Project Objectives:} Due to the scale and complexity of
the next generation of physical and biological experiments we require
new techniques and algorithms that can characterize and classify data
in a way that is inherently probabilistic and data-centric.  The goal
of this proposal is to approach this challenge through the development of
new methodologies, centered around the theme of active learning, that
can optimize the scientific returns of petascale experiments. Our
primary objectives are: the design of techniques using Gaussian
processes that are robust to missing data and non-Gaussian errors; the
development of active learning and active feature selection techniques
that can identify the next measurement or observation to make in order 
improve the overall classification or calibration of a data set; and the application
of these techniques to cosmological surveys in order to improve our
constraints of dark energy.

\noindent{\bf Project Description:} A new generation of DOE sponsored
data intensive experiments and surveys, designed to address
fundamental questions in physics, materials and biology, will come
on-line over the next decade. These experiments share a set of common
challenges: how do we choose the next experiment or observation to
maximize our scientific returns; how do we identify anomalous
sources that may be indicative of new classes of object from a
continuous stream of data; and how do we characterize and classify
events within data streams that are inherently noisy and
incomplete. This proposal addresses these challenges through the
development of a broad class of novel and scalable machine-learning
techniques centered around the theme of active learning.

Active learning algorithms iteratively decide which data points they
will collect outputs on and add to a training set.  Their goal is to
choose the points that will most improve the model being learned.  At
each step, they consider the current training data, the potential data
that might be obtained, and the current learned model, and evaluate
what would be the best choice for the next observation, experiment, or
feature such that it improves our knowledge of the overall system
(according to some objective criterion).  In so doing they can
optimize the speed and performance of a classification algorithm over
simpler sampling strategies.

The algorithms and methodologies we propose to develop here will
initially focus on DOE sponsored cosmology experiments (i.e.\ the Dark
Energy Survey, and the Large Synoptic Survey Telescope). These surveys
are ideal proxies as their bandwidth (terabytes of data per night and
petabytes of data every couple of months) will enable high precision
studies impacting the understanding of cosmology, particle physics,
and potentially theories of gravity.  Our ability to achieve these
scientific goals relies on analyses at a scale, speed, and complexity
beyond the capabilities of current automated machine learning methods.

\noindent{\bf Project Outcomes:}
The potential impact of active learning algorithms that scale to the
size and complexity of petascale data sets is substantial. The
development of our proposed techniques will enable the optimization of
the scientific returns from billion dollar investments in
observational facilities. Achieving these breakthroughs will enable
advances and new methodologies for scaling active learning that will
be applicable to experiments from physics, to biology, to climatology.

\end{document}
